\section{Properties of Policies}
\label{sec:properties-policies}

In this section we will 
show that properties of policies can be 
expressed based on several partial ordering relations.
For example, we might want to prove
that a (possibly very complex) policy at least protects as much as
some simpler policy, and similarly we might want to guarantee that a
(possibly very complex) policy does not say anything outside of its
scope. Such properties can be expressed using the ordering relations
defined below.

Let $P_1 = \langle S_1, R_1, T_1 \rangle$ and let $P_2 =
\langle S_2, R_2, T_2 \rangle$ be two policies.  
We define the following
partial orders:
\begin{eqnarray*}
  P_1 \sqsubseteq_P P_2 & \mbox{iff} & S_1 \subseteq S_2 \\
  P_1 \sqsubseteq_D P_2 & \mbox{iff} & R_1 \subseteq R_2 \\
  P_1 \sqsubseteq_E P_2 & \mbox{iff} & T_1 \subseteq T_2 \\
  P_1 \sqsubseteq_{P,D,E} P_2 & \mbox{iff} & P_1 \sqsubseteq_P P_2
  \wedge P_1 \sqsubseteq_D P_2 \wedge P_1 \sqsubseteq_E P_2 
\end{eqnarray*}
Note that, we can define a partial order for
for any combination of of $P$, $D$ and $E$.
We define $P_1 \sqsubseteq P_2 \equiv P_1 \sqsubseteq_{P,D,E} P_2$.
We can regard $P_1 \sqsubseteq P_2$ as stating that for any $e \in E$
where $\eff (P_1, e) \neq \NotApplicable$, $\eff (P_2, e) = \eff (P_1,
e)$.

To demonstrate the use of these ordering relations, 
let us create a new policy; people are permitted
to check the current results of the election, for exit polls.  
We encode this with the
following policy 
\begin{eqnarray*}
  S_5 & = & \{ \langle a, v, o \rangle \in E : 
  \exists x \in o \; x = \mtt{getresult} \} \\
  r_3 & = & \Scope (\Err (\Permit, S_4), S_5)
\end{eqnarray*}
where $S_4$ is defined in Equation~\eqref{eq:s4}.
Now, we can create a composite policy as follows $p_c = p \oplus r_3$,
where $p$ is defined in Equation~\eqref{eq:p}.  This
policy has a bug---specifically, it permits people under 18 to vote in
certain circumstances---and we will demonstrate the usefulness of our
technique by showing this.  First, we perform our translations on this
new policy as above, getting:
\begin{eqnarray*}
  g(f(r_3)) & = & \langle S_5 \setminus S_4, \emptyset, S_4 \cap S_5 \rangle \\
  g(f(p_c)) 
    & = &  \langle ((S_2 \setminus (S_0 \cup S_1 \cup S_3 \cup S_4)) \\
    & & \quad \cup (S_5 \setminus S_4)) \setminus
    (((S_3 \cup S_4) \cap S_2) 
    \cup (S_4 \cap S_5)), \\
    & & (((S_0 \cup S_1) \setminus (S_3 \cup S_4)) \cap S_2) \\
    & & \quad \setminus ((S_5 \setminus S_4)
    \cup (S_4 \cap S_5)), \\
    & & ((S_3 \cup S_4) \cap S_2) \cup (S_4 \cap S_5) \rangle
\end{eqnarray*}
where $S_0,S_1, S_3$ and $S_4$ are from
Equations~\eqref{eq:s1} to ~\eqref{eq:s4}.
Using set algebra we can simplify the expression for policy $p_c$ to
\begin{align*}
  g(f(p_c)) = \langle & \left( S_2 \setminus (S_0 \cup S_1 \cup S_3 \cup S_4) \right)
    \\ & \quad
    \cup \left( S_5 \setminus ((S_3 \cap S_2) \cup S_4)
    \right), \\
    & ((S_0 \cup S_1) \cap S_2) \setminus (S_3 \cup S_4 \cup
    S_5), \\
    & ((S_3 \cup S_4) \cap S_2) \cup (S_4 \cap S_1 \cap S_5) \rangle
\end{align*}
Now, we insist that this combined policy deny anyone trying to vote
who is under 18.  This is itself a policy, which we call $p_v$:
\begin{align*}
  p_v = \langle \emptyset, (S_0 \cap S_2) \setminus (S_3 \cup S_4),
  (S_3 \cup S_4) \cap S_2 \rangle
\end{align*}
The interesting thing here is whether or not $p_v \sqsubseteq_D p_c$,
i.e., does the policy $p_c$ deny every input that is denied by $p_v$.
That would mean that everyone trying to vote who is under 18 is
denied, and that our policy combination has not done any harm.
However, the environmental tuple \[e = \langle \{17\}, \{\mtt{true}\},
\{\mtt{vote}, \mtt{getresult} \} \rangle\] demonstrates that that is
not the case.  Input $e$ passes the second part of the \Permit{}
requirement and so is permitted by $p_c$ (which means that it is {\em
  not} denied by $p_c$) but denied by $p_v$, i.e., $e$ demonstrates
that $p_v \not \sqsubseteq_D p_c$.  The error is that, we do not
enforce that only one action be given in the third component of the
input, and because of this we have the surprising result that someone
who is under eighteen and has already voted, but asks for the voting
results at the same time as trying to vote will be permitted, and so
can cast any number of ballots.  To fix this, we could insist upon a
new condition, that $\exists ! \, x \in o$; or we could use $\otimes$
instead of $\oplus$, which would ensure that only one of the
sub-policies could be definitive on any given point (and so turn $\eff
(e, p_v)$ into an \Indeterminate{} result instead of a \Permit{}); or we
could decide that only people who have voted already can check the
results.

% arch-tag: e2b0e59e-49da-4b4c-9860-1da5c30b891f

%%% Local Variables: 
%%% mode: latex
%%% TeX-master: "paper"
%%% End: 
