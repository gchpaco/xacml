
\newcommand{\Union}{\bigcup}
\newcommand{\union}{\cup}
\newcommand{\Intersect}{\bigcap}
\newcommand{\intersect}{\cap}
\renewcommand{\And}{\wedge}
\newcommand{\Or}{\vee}
\newcommand{\band}{\bigwedge}
\newcommand{\bor}{\bigvee}
\newcommand{\Implies}{\Rightarrow}
\newcommand{\Iff}{\Leftrightarrow}
\renewcommand{\implies}{\rightarrow}
\renewcommand{\iff}{\leftrightarrow}
\newcommand{\true}{\texttt{true}}
\newcommand{\false}{\texttt{false}}
\newcommand{\assign}{:=}

\section{Automatically Proving Properties of Policies}
\label{sec:autom-prov-prop}

Given the formal model defined in Section~\ref{sec:formal-model} and
properties defined in Section~\ref{sec:properties-policies} we would
like to check properties of access policies automatically.  To do this
we first formalize the syntax of formulas we use to specify subsets of
$E$.  Then we discuss how policies constructed using these formulas
and policy combinators can be translated to Boolean logic formulas.
After this translation we show that we can check properties of access
policies using a SAT solver.

\subsection{Characterizing Subsets of the Environment}
\label{sec:policy-classes}

In Section~\ref{sec:formal-model}, we defined our formal model using
subsets of the set of possible environments $E$.  We showed that each
policy can be expressed in triple form $P= \langle S, R, T \rangle$
where $S$, $R$, and $T$ are subsets of $E$.  We will assume that all
subsets of $E$ are specified in the form:
\[ 
\{ e \in E : C \}
\]
where $C$ is a constraint that evaluates to true or false for each
environment, i.e., the only free variables in $C$ are the components
of the environment tuple $e$.  Note that the sets $S_0, S_1, \dots,
S_4$ in Equation~\eqref{eq:s0} are expressed this way.

Given a set in the form $S = \{ e \in E : C \}$ our goal is to
generate a boolean logic formula $B$ which encodes the set $S$.  The
encoding will map each $e \in E$ to a valuation of the boolean
variables in $B$, and $B$ will evaluate to true if and only if $e \in
S$.  Based on such an encoding we can convert questions about
different policies (such as if one subsumes the other one) to SAT
problems and then use a SAT solver to check them.  For example, we can
generate a boolean formula which is satisfiable if and only if an
access policy is not subsumed (i.e., $\not \sqsubseteq$) by another
one.  If the SAT solver returns a satisfying assignment to the
formula, then we can conclude that the property is false, and generate
a counterexample based on the satisfying assignment.  If the SAT
solver declares that the formula is not satisfiable then we can
conclude that the property holds.  We will discuss the details of such
a translation below.

\begin{figure*}[t]
\[
\begin{array}{lll}
1 & SCAL \rightarrow  A &
        SCAL.f \assign SCAL.v[A]
        \And \band_{i=1, i \neq A}^k (\neg SCAL.v[i]) \\
2 & SCAL \rightarrow  a &
        SCAL.f \assign \band_{i=1}^k (SCAL.v[i] \iff a[i])
        \And (\bor_{i=1}^k SCAL.v[i]) \And \band_{i=1}^k
        (SCAL.v[i] \implies \band_{j=1,j\neq i}^k \neg SCAL.v[j])  \\
3 & BSET \rightarrow  s &
        BSET.f \assign \band_{i=1}^k (BSET.v[i] \iff s[i]) \\
4 & BSET \rightarrow  e[i] &
        BSET.f \assign \band_{j=1}^k (BSET.v[j] \iff e[i][j]) \\
5 & SET  \rightarrow  \{ SCAL \} &
        SET.f \assign SCAL.f \And \band_{i=1}^k (SET.v[i] \iff SCAL.v[i]) \\
6 & SET  \rightarrow  BSET &
        SET.f \assign BSET.f \And \band_{i=1}^k (SET.v[i] \iff BSET.v[i]) \\
7 & SET  \rightarrow  SET_1 \cup SET_2 &
        SET.f \assign SET_1.f \And SET_2.f \And
        \band_{i=1}^k (SET.v[i] \iff (SET_1.v[i] \Or SET_2.v[i])) \\
8 & SET  \rightarrow SET_1 \cap SET_2 &
        SET.f \assign SET_1.f \And SET_2.f \And
        \band_{i=1}^k (SET.v[i] \iff (SET_1.v[i] \And SET_2.v[i])) \\
9 & SET  \rightarrow SET_1 \setminus SET_2 &
        SET.f \assign SET_1.f \And SET_2.f \And
        \band_{i=1}^k (SET.v[i] \iff (SET_1.v[i] \And \neg SET_2.v[i])) \\
10 & BP \rightarrow  \true &
        BP.f \assign BP.b \iff \true \\
11 & BP \rightarrow  \false &
        BP.f \assign BP.b \iff \false \\
12 & BP \rightarrow  SCAL_1 = SCAL_2 &
        BP.f \assign SCAL_1.f \And SCAL_2.f \And
        (BP.b \iff \band_{i=1}^k (SCAL_1.v[i] \iff SCAL_2.v[i])) \\
13 & BP \rightarrow  SCAL \in SET &
        BP.f \assign SCAL.f \And SET.f \And
        (BP.b \iff \band_{i=1}^k (SCAL.v[i] \implies SET.v[i])) \\
14 & BP \rightarrow  SET_1 \subseteq SET_2 &
        BP.f \assign SET_1.f \And SET_2.f \And
        (BP.b \iff \band_{i=1}^k (SET_1.v[i] \implies SET_2.v[i]))
\end{array}
\]
\caption{Translation of the basic predicates to Boolean logic formulas.}
\label{fig:BP}
\end{figure*}

To present our translation we use the following notational
conveniences: for elements $e \in E$, we name the components of $e$
$e[0]$, $e[1]$, $\dots$, $e[n]$.  We use $s$, $s_0$, $s_1$, $\dots$,
$s_n$ to denote set variables, $a$, $a_0$, $a_1$, $\dots$, $a_n$ to
denote scalar variables, and $A$, $A_0$, $A_1$, $\dots$, $A_n$ to
denote constants.  Finally, $BP$ is a set of basic predicates which we
define as follows:
\begin{eqnarray*}
  SCAL & \rightarrow & A \quad | \quad a \\
  BSET & \rightarrow & s \quad | \quad e[i] \\
  BP & \rightarrow &  \texttt{true} \quad | \quad \texttt{false} \\
 & | & SCAL = SCAL \quad | \quad SCAL \in SET  \\
 & | &  SET \subseteq SET \\
  SET & \rightarrow & BSET  \quad | \quad \{ SCAL \} \\
  & | & SET \cup SET \quad | \quad SET \cap SET  \\
  & | &  SET \setminus SET \\
\end{eqnarray*}
The above grammar is sufficient for specifying policies using only
enumerated types (which obviously have finite domains) and the simple
operations $\neg$, $=$, $\in$, $\subseteq$.  We will discuss extension
to other domains later in this section.

Assuming that all subsets of $E$ are specified in the form $\{ e \in E
: C \}$, where there are no free variables save $e$ in $C$, $C$ is
defined as follows:
\begin{eqnarray*}
  C & \rightarrow & BP \quad | 
      \quad C \wedge C \quad | \quad C \vee C \quad | \quad \neg C \\
 & | &   \forall a \in BSET \; C \quad |
  \quad \exists a \in BSET \; C  \\
  & | & \exists ! \, a \in BSET \; C \quad 
\end{eqnarray*}
We use $\exists !$ to mean there exists exactly one instance that
holds.  We can express all set definitions on unordered and enumerated
types that are permitted in XACML using the expressions above.

We will explain our translation of a constraint $C$ defined by the
above grammar to a Boolean logic formula using attribute grammars.  We
will first discuss the translation of the basic predicates $BP$.  In
order to simplify our presentation we will assume that domains of all
the scalars have the same size, call it $k$.  We will encode a set of
values from any domain using a Boolean vector of size $k$.  Given a
Boolean vector $v$, we will denote its components as $v[1]$, $v[2]$,
$\dots$, $v[k]$ where $v[i] \iff \true$ means that element $i$ is a
member of the set represented by $v$ whereas $v[i] \iff \false$ means
that it is not.  We encode a set variable $s$ and each component of
the environment tuple $e$ using the same encoding, i.e., as a vector
of Boolean values.  To simplify our presentation we also encode a
scalar variable $a$ as a set using a vector of Boolean values but
restrict it to be a singleton set by making sure that at any time only
one of the Boolean values in the vector can be true.  In our actual
implementation scalar variables are represented using $\log_2 k$
Boolean variables where $k$ is the size of the domain.

The attribute grammar for basic predicates is shown in
Figure~\ref{fig:BP}.  We have numbered the production rules.  Each
production rule has a corresponding semantic rule next to it.
Semantic rules describe how to compute the attributes of the
nonterminal on the left hand side of the production rule using the
attributes of the terminals and nonterminals on the right hand side of
the production rule.  In the attribute grammar shown in
Figure~\ref{fig:BP} the nonterminals $SCAL$, $BSET$ and $SET$ have two
attributes.  One of them is a Boolean vector $v$ denoting a set of
values, and the other one is a Boolean logic formula $f$ which
accumulates the frame constraints.  Again to simplify our presentation
we represent scalar constants and scalar variables (i.e., the
non-terminal $SCAL$) as singleton sets whereas in our actual
implementation they are represented using $\log_2 k$ Boolean
variables.

\begin{figure*}[t]
\[
\begin{array}{lll}
1 & C \rightarrow  BP &
        C.f \assign BP.f \And
        (C.b \iff BP.b) \\
2 & C \rightarrow  \neg C_1 &
        C.f \assign C_1.f \And
        (C.b \iff \neg C_1.b) \\
3 & C \rightarrow  C_1 \vee C_2 &
        C.f \assign C_1.f \And C_2.f \And
        (C.b \iff (C_1.b \Or C_2.b)) \\
4 & C \rightarrow  C_1 \wedge C_2 &
        C.f \assign C_1.f \And C_2.f \And
        (C.b \iff (C_1.b \And C_2.b)) \\
5 & C \rightarrow  \forall a \in BSET \; C_1  &
        C.f \assign BSET.f \And C_1.f \And
        (\band_{i=1}^k (BSET.v[i] \implies
         (a[i] \And \band_{j=1,j \neq i}^k \neg a[j] \And C_1.b))) \\
6 & C \rightarrow  \exists a \in BSET \; C_1  &
        C.f \assign  BSET.f \And C_1.f \And
        (\bor_{i=1}^k (BSET.v[i] \implies
        (a[i] \And \band_{j=1,j \neq i}^k \neg a[j] \And C_1.b))) \\
7 & C \rightarrow  \exists ! \, a \in BSET \; C_1  &
  C.f \assign BSET.f \And C_1.f \And
  (\bor_{i=1}^k (BSET.v[i] \implies
   (a[i] \And \band_{j=1,j \neq i}^k \neg a[j] \And C_1.b))) \\
   & & \And (\band_{i=1}^k ((BSET.v[i] \And
   a[i] \And \band_{j=1,j \neq i}^k \neg a[j] \And C_1.b) 
   \implies \neg \bor_{l=1, l \neq i}^k (BSET.v[l] \And
   a[l] \And \band_{j=1,j \neq l}^k \neg a[j] \And C_1.b)))
\end{array}
\]
\caption{Translation of the constraints to Boolean logic formulas.}
\label{fig:C}
\end{figure*}

Rule~1 in Figure~\ref{fig:BP} states that a scalar constant $A$ is
encoded as a singleton set that contains only $A$.  This singleton set
is represented as a Boolean vector $v$, such that $v[A]$ is set to
true and all the rest of the elements of the vector are set to false.
This condition is stored in the frame constraint $f$.  Rule~2 states
that a scalar variable is also encoded as a Boolean vector $v$.  The
frame constraint $f$ makes sure that the elements of the Boolean
vector $v$ are same as the elements of the Boolean vector representing
the scalar variable $a$ and exactly one of the elements in $a$ or $v$
is set to true in any given time.  Rules~3 and~4 show that the set
variables ($s$) and components of the environment tuple ($e[i]$) are
also encoded as Boolean vectors.

Rule~5 creates a singleton set from a scalar constant $SCAL$.
However, since we encode scalar constants as singleton sets, this
simply means that the Boolean vectors encoding the scalar constant
($SCAL.v$) and the set ($SET.v$) are equivalent and the frame
constraint $SET.f$ expresses this constraint.  Note that in the
attribute grammar shown in Figure~\ref{fig:BP} the frame constraint of
a nonterminal on the left hand side of a production is a conjunction
of the frame constraints of the nonterminals on the right hand side of
the production plus some other constraints that are added based on the
production rule.

Rules~7, 8 and~9 encode the set operations: union, intersection and
set difference.  Each set operation on two set expressions $SET_1$ and
$SET_2$ results in the creation of new Boolean vector $SET.v$.  The
value of an element in $SET.v$ is defined based on the corresponding
elements in $SET_1.v$ and $SET_2.v$.  For example for the union
operation $SET.v[i]$ is true if and only if $SET_1.v[i]$ is true or
$SET_2.v[i]$ is true.  The intersection and set difference are defined
similarly.

The nonterminal $BP$ corresponds to the basic predicates and it has
two attributes. One of them is a boolean variable $b$ representing
the truth value of the predicate and the other one is a
Boolean logic formula $f$ that accumulates the frame constraints.

Rules~10 and~11 create two basic predicates which have the truth value
true and false, respectively.

Rule~12 is a basic predicate that corresponds to an equality
expression comparing two scalars.  Since scalars are expressed as
Boolean vectors, the Boolean variable encoding the truth value of the
predicate is true if and only if all elements of the Boolean vectors
encoding the two scalar values are the same.  This constraint is added
to the frame constraint of the basic predicate.

Rule~13 creates a basic predicate that corresponds to a membership
expression testing membership of a scalar to a set expression.
Rule~14 creates a basic predicate that corresponds to a subset
expression testing if a set expression is subsumed by another set
expression.  Since we encode scalars a singleton sets, the frame
constraints generated for rules~13 and~14 are very similar.  They
state that if a value is a member of the set on the left hand side,
then it should also be member of the set on the right hand side.

The attribute grammar for the constraints is shown in
Figure~\ref{fig:C}.  The nonterminal $C$ has two attributes.  One of
them is a boolean variable $b$ representing the truth value of the
constraint, and the other one is a Boolean logic formula $f$ that
accumulates the frame constraints.  Again, the frame constraint of a
nonterminal on the left hand side of a production is a conjunction of
the frame constraints of the nonterminals on the right hand side of
the production plus some other constraints that are added based on the
production rule.

Rule~1 is just a syntactic rule expressing that a constraint can be a
basic predicate.  Rule~2 defines the negation operation.  As expected
the frame constraint states that the value of the constraint on the
left hand side of the production rule is the negation of the value of
the constraint on the right hand side of the production rule.  Rules~3
and~4 define the disjunction and conjunction operations.  The frame
constraints generated in Rules~3 and~4 state that the value of the
constraint on the left hand side of the production rule is the
disjunction or the conjunction of the values of the constraints on the
right hand side of the production rule, respectively.

Rules~5, 6 and~7 deal with quantified constraints.  In the rules~5, 6
and~7, $a$ denotes a scalar variable which is quantified over a basic
set expression $BSET$ which is either a set variable $s$ or a
component of the environment tuple $e[i]$.  The quantified variable
$a$ can appear as a free variable in the constraint expression on the
right hand side ($C_1$).  Universal quantification is expressed as a
conjunction which states that for all the members of the set $s$ or
$e[i]$, the constraint $C_1$ should evaluate to true.  This is
achieved by restricting the value of the scalar variable $a$ to the
value of a different member of the set for each conjunct.  Existential
quantification is expressed similarly as a disjunction by restricting
the value of the scalar variable $a$ to the value of a different
member of the set for each disjunct.

Rule~7 is an existentially quantified constraint which evaluates to
true if and only if the constraint $C_1$ evaluates to true for exactly
one member of the set $s$ or $e[i]$.  This is expressed by first
stating that there is at least one member of the set $s$ or $e[i]$ for
which the constraint $C_1$ evaluates to true (which is equivalent to
existential quantification) and then adding an extra conjunction which
states that the constraint $C_1$ does not evaluate to true for two
different members of the set $s$ or $e[i]$.

The translation we described above can handle XACML policies which
only use bounded unordered and enumerated types.  In fact, during our
analysis we limit the size of every domain to a given fixed size and
then analyze the policies as though they were specified using finite
enumerated sets of that size.  The problem is that if our automated
analysis does not yield a counterexample to a given property, then
that does not necessarily mean that no counterexample exists---perhaps
if we had increased the scope just a little more we would have found
one.  The small scope hypothesis (discussed
in~\cite{jackson:elements}, and tested and confirmed for some data
structure algorithms in~\cite{marinov:exhaustive}) suggests that small
scopes could be sufficient in practice.  Note that if a counterexample
is found, that counterexample is definite and can be translated into
an error in the original policy.

Finally, the translation we described above does not handle domain
specific predicates, e.g., ordering relations on types such as
integers.  When we translate sets described using such predicates to
boolean logic formulas we represent them as uninterpreted Boolean
functions.  We create a Boolean variable for encoding the value of the
uninterpreted boolean function and we generate constraints which
guarantee that the value of the function is the same if its arguments
are the same.  Other than this restriction the variables encoding the
functions can get arbitrary values.  Note that this brings an extra
level of imprecision to our analysis.  We were not able to trust the
positive results because of the scope restriction, but now it is also
possible that counterexamples may be spurious, and will need to be
validated against the original policy.  However, we think that such
automated analysis can still be useful in uncovering errors in access
policies.

Note that, it is possible to fully interpret ordering relations.  We
can encode a type with a domain of $n$ ordered elements using $n^2$
boolean variables, one for each pair of values in the domain,
representing the ordering relations.  However, XACML uses many complex
functions such as XPath matching and X500 name matching which can be
lead to very complex formulas if one tries to fully interpret them in
the Boolean logic translation.  Hence, we believe that using
uninterpreted functions for abstracting such complex functionality is
a justified approach which enables us to handle a significant portion
of the XACML language.  Also, we would like to note that the
imprecision caused by abstraction of such complex functions did not
lead to any spurious results in the experiments we performed so far.

\subsection{Verification of Policies}

As discussed in Section~\ref{sec:properties-policies}, we specify
properties of policies using a set of partial ordering relations.
These partial ordering relations can be used to state that a certain
type of outcome for one policy subsumes the same type of outcome for
another policy.  In this section we will only focus on the
$\sqsubseteq$ relation.  Translation of properties specified using
other relations are handled similarly.

Given a query like $P_1 \sqsubseteq P_2$, our goal is to generate a
Boolean logic formula which is satisfiable if and only if $P_1 \not
\sqsubseteq P_2$.  As we discussed earlier our tool first translates
policies $P_1$ and $P_2$ to triple form, such that $P_1 = \langle S_1,
R_1, T_1 \rangle$ and $P_2 = \langle S_2, R_2, T_2 \rangle$ where each
element of each triple is specified with a constraint expression as
follows:
\begin{eqnarray*}
S_1  & = & \{ e \in E : C_{S_1} \} \\
R_1  & = & \{ e \in E : C_{R_1} \} \\
T_1  & = & \{ e \in E : C_{T_1} \} \\
S_2  & = & \{ e \in E : C_{S_2} \} \\
R_2  & = & \{ e \in E : C_{R_2} \} \\
T_2  & = & \{ e \in E : C_{T_2} \}
\end{eqnarray*}

After translating policies $P_1$ and $P_2$ in to the triple form our
translator generates a boolean logic formulas for the constraints
$C_{S_1}$, $C_{R_1}$, $C_{T_1}$, $C_{S_2}$, $C_{R_2}$ and $C_{T_2}$
based on the attribute grammar rules described in Figures~\ref{fig:BP}
and~\ref{fig:C}.  For example, after this translation the truth value
of the constraint $C_{S_1}$ is represented with the Boolean variable
$C_{S_1}.b$ and the frame constraint $C_{S_1}.f$ states all the
constraints on the Boolean variable $C_{S_1}.b$.

Recall that, given $P_1 = \langle S_1, R_1, T_1 \rangle$ and $P_2 =
\langle S_2, R_2, T_2 \rangle$, $P_1 \sqsubseteq P_2$ holds if and
only if
\[
  S_1 \subseteq S_2
  \wedge R_1 \subseteq R_2 \wedge T_1 \subseteq T_2
\]
Based on this, we generate a formula $F$ such that 
\[
F = \true \ \text{ iff } \ P_1 \sqsubseteq P_2
\]
as follows:
\begin{eqnarray*}
F & = & (C_{S_1}.f \wedge C_{R_1}.f \wedge C_{T_1}.f \wedge 
    C_{S_2}.f \wedge C_{R_2}.f \wedge C_{T_2}.f ) \implies \\
  & &   ((C_{S_1}.b \implies C_{S_2}.b) \wedge (C_{R_1}.b \implies C_{R_2}.b)
    \wedge (C_{T_1}.b \implies C_{T_2}.b))
\end{eqnarray*}

Finally, we send the property $\neg F$ to the SAT solver.  If the SAT
solver returns a satisfying assignment for the Boolean variables
encoding the environment tuple $e$ (which are the only free variables
in the formula $\neg F$), the satisfying assignment corresponds to a
counter-example environment demonstrating how the property is
violated.  If the SAT solver states that $\neg F$ is not satisfiable,
then we conclude that the property holds, i.e., $P_1 \sqsubseteq P_2$.

Since majority of the SAT solvers expect their input to be expressed
in Conjunctive Normal Form (CNF), the last step in our translation
(before we send the formula $\neg F$ to the SAT solver) is to convert
$\neg F$ to CNF.  For conversion to CNF we have implemented the
structure preserving technique from \cite{plaisted86structure}.

% arch-tag: 931ea1c1-9ebf-4e94-89e7-06ac1ad3856f

%%% Local Variables: 
%%% mode: latex
%%% TeX-master: "paper"
%%% End: 
