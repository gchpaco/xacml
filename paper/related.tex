\section{Related Work}
Access control has been extensively researched:
\cite{samarati01policiesmodelsmechanisms,sandhu94access,sandhu96audit}~introduce
the process,
\cite{bertino98access,bonatti01dataarchives,sandhu93latticebased,sandhu96rolebased}~describe
various models for access control,
\cite{damiani02SVG,damiani01controllingaccess,damiani02fine,damiani00design}~describe
a particular fine grained access control for XML documents,
\cite{bonatti02algebra}~defines an algebra for composing different
parts of a model into a unified whole and
\cite{abadi93calculus,heckman98applying,divimercati96federated}~speak
of distributing the control so that it is consistent across a
distributed system.

Access policy languages, too, are not new:
\cite{abadpeiro99plas}~describes a general purpose policy language for
authorization systems, \cite{jajodialogical}~defines a model and
language for access control and \cite{jajodia01multiplepolicies}~and
\cite{jajoida97unified} present a framework for enforcing multiple
access policies by expressing how to combine them in a new language.
We chose XACML because it is a standardized language with tool
support, and so our results are more likely to be immediately useful.

The problem with access policies becoming large and difficult to
reason about has also been studied, but not in the general case:
\cite{heckman98applying}~speaks of verifying a hierarchy of security
servers to ensure that they are enforcing the whole access policy,
and \cite{naumovich:permission-analysis}~presents an algorithm for
computing the flow of permissions through the Java security model, to
aid static analysis.  Neither of these are exactly what we want:
\cite{heckman98applying}~can prove that the programs you have
collectively implement the policy you specified, but their technique
cannot tell you whether you have made a subtle error in creating your
policy in the first place; \cite{naumovich:permission-analysis} is
more comprehensive but is specific to Java's security model.

Automated analysis of access control policies has also been
researched; \cite{schaad:lightweight}~and \cite{zao:rbac}~analyze role
based access control schemas using the Alloy analyzer.
However \cite{schaad:lightweight}~uses Alloy to verify that
the composition of specifications is well formed and is silent about
their content, whereas we introduce a formal model of and a partial
ordering on XACML specifications specifically designed for analyzing
the semantics.
\cite{zao:rbac}~models RBAC schema in Alloy and then
checks these models against predicates, also written in Alloy.
We introduce a formal model for XACML with a partial ordering on policies
that we then automatically check using a SAT solver as a back end; we do not
insist that the user write predicates in another language and operate
solely on XACML.

Alloy analyzer also uses a SAT solver as a back-end to solve verification
queries \cite{jackson00automating,jackson:alcoa}. 
Hence, translating XACML policies to Alloy in order to verify them
is in effect an indirect way of using a SAT solver for verification.
In fact, we also tried using Alloy analyzer for verification of XACML policies \cite{hughes04}.
However our experiments showed that a direct
translation to SAT is much more effective then translating
the verification queries to Alloy. 
Using a direct translation we can generate a customized encoding 
of the problem. Alloy analyzer is optimized for a more general class
of models and hence, not necessarily efficient for types of verification
queries we are interested in.

Recently, Fisler et al. \cite{fisler05} used multi-terminal decision
diagrams to verify properties of XACML policies. Verification
queries in \cite{fisler05} are expressed differently than the approach we use.
In our approach verification queries are stated as relationships between
policies which makes our tool very easy to use. 
A verification approach that uses decision diagrams is more likely to be
successful for incremental analysis techniques. Hence we believe that
the multi-terminal decision
diagrams are the appropriate representation to use for the change-impact analysis
presented in \cite{fisler05}. 
However for the type of verification queries we discuss in this paper we expect a verification
approach based on SAT solvers to
perform better than a verification approach based on decision diagrams. 
Our experimental results support this view.
