\documentclass{acm_proc_article-sp}
\usepackage{amsmath}
\usepackage{times}
\usepackage{mathptm}
\renewcommand{\ttdefault}{cmtt}
\DeclareMathOperator{\eff}{eff}
\DeclareMathOperator{\xpathmatch}{xpathnodematch}
\DeclareMathOperator{\translation}{translate}
\DeclareMathOperator{\extract}{extract}
\DeclareMathOperator{\constant}{constant}
\DeclareMathOperator{\env}{env}
\newcommand{\step}[1]{\intertext{\{#1\}}}
\newcommand{\mtt}[1]{\text{\texttt{#1}}}
\newcommand{\itmathcommand}[1]{\ensuremath{\text{\textit{#1}}}}
\newcommand{\Scope}{\itmathcommand{Scope}}
\newcommand{\Err}{\itmathcommand{Err}}
\newcommand{\Permit}{\itmathcommand{Permit}}
\newcommand{\Deny}{\itmathcommand{Deny}}
\newcommand{\Indeterminate}{\itmathcommand{Indet}}
\newcommand{\NotApplicable}{\itmathcommand{NotApp}}

\begin{document}
\title{Automated Verification of Access Control Policies}
\numberofauthors{2}
\author{
\alignauthor Graham Hughes \quad Tevfik Bultan \\
      \affaddr{Computer Science Department} \\
      \affaddr{University of California} \\
      \affaddr{Santa Barbara, CA 93106, USA} \\
      \email{\{graham,bultan\}@cs.ucsb.edu} \\
}
\date\today
\maketitle

\begin{abstract}
  Managing access control policies in modern computer systems can be
  challenging and error-prone, especially when multiple access
  policies are combined to form new policies, possibly introducing
  unintended consequences.  In this paper we present a framework for
  automated verification of access control policies.  We introduce a
  formal model for systematically specifying access to resources.  We
  show that the access control policies in the XACML access control
  language can be translated to a simple form which partitions the
  input domain to four classes: permit, deny, error, and
  not-applicable.  We present several ordering relations for access
  control policies which can be used to specify the properties of
  the policies and the relationships among them.  We then show how to
  automatically check these ordering relations using an existing
  automated analysis tool.  In particular, we translate XACML policies
  to the Alloy language and check their properties using the Alloy
  Analyzer.  Our experimental results demonstrate that automated
  verification of XACML policies is feasible.
\end{abstract}


\section{Introduction}

Keeping track of permission grants, organizational policies and
special cases in a modern software system has become difficult enough
on its own; keeping the policy consistent across multiple
heterogeneous systems is even more difficult, as each system requires
that these grants and policies be expressed in its own specific
control language.  Several unified access policy languages address
this problem.  If all the systems use the same access policy language,
then access policies need only be written once and similarly only one
policy needs to be kept up to date.  In this paper we focus on one
particular such language, the OASIS standard XACML~\cite{xacml}.

OASIS (the Organization for the Advancement of Structured Information
Standards) is an international standards consortium noted particularly
for work in the popular markup language XML~\cite{XML}.  They have
published a standard for access policy languages called XACML
(expanded as ``eXtensible Access Control Markup Language''), which we
will discuss in detail below.  For the moment, it suffices to describe
it as an XML-based language for expressing access rights to arbitrary
objects that are identified in XML, with a particular focus on the
composition of many different subpolicies into a single superpolicy.

Having such a combined policy is convenient, but such a policy will
inevitably become quite large and complex as all an organization's
rules get placed in it.  It is possible, even likely, that the act of
creating a unified policy out of numerous disparate smaller policies
could leave it vulnerable to unintended consequences.  In this paper
we investigate statically verifying properties of access control
policies to prevent such errors. We translate XACML policies into a
simplified mathematical model, which we reduce to a normal form
separating the conditions that give rise to {\em access permitted},
{\em access denied}, and {\em internal error} results. We define
partial orderings between access control policies, with the intention
of checking whether a policy is over- or under constrained with
respect to another one.  We show that these ordering relations can be
translated to logical expressions which evaluate to true if and only
if the corresponding relation holds.  We use Alloy analyzer to check
the truth value of these logical expressions automatically. Using our
translator and Alloy analyzer, we can check if a combination of XACML
policies does or does not faithfully reproduce the properties of its
subpolicies, and thus discover unintended consequences before they
appear in practice.

In Section~\ref{sec:policy-spec} we develop a formal model for access
policies; in Section~\ref{sec:transformations} we discuss how to
transform these models into a normal form that distinguishes access
permitted, access denied, and error conditions.  In
Section~\ref{sec:properties-policies} we define partial ordering
relations among access policies which are used to specify their
properties.  We show how to check these properties automatically in
Section~\ref{sec:autom-prov-prop}.  Specifically, we discuss a
hierarchy of access policies in Section~\ref{sec:policy-classes} which
we use in Section~\ref{sec:translation-alloy} to show how to translate
a policy to the Alloy modeling language.  Finally, we report the
results of our experiments in Section~\ref{sec:experiments}.

{\bf Related Work:}
Access control has been extensively researched:
\cite{samarati01policiesmodelsmechanisms,sandhu94access,sandhu96audit}~introduce
the process,
\cite{bertino98access,bonatti01dataarchives,sandhu93latticebased,sandhu96rolebased}~describe
various models for access control,
\cite{damiani02SVG,damiani01controllingaccess,damiani02fine,damiani00design}~describe
a particular fine grained access control for XML documents,
\cite{bonatti02algebra}~defines an algebra for composing different
parts of a model into a unified whole and
\cite{abadi93calculus,heckman98applying,divimercati96federated}~speak
of distributing the control so that it is consistent across a
distributed system.

Access policy languages, too, are not new:
\cite{abadpeiro99plas}~describes a general purpose policy language for
authorization systems, \cite{jajodialogical}~defines a model and
language for access control and \cite{jajodia01multiplepolicies}~and
\cite{jajoida97unified} present a framework for enforcing multiple
access policies by expressing how to combine them in a new language.
We chose XACML because it is a standardized language with tool
support, and so our results are more likely to be immediately useful.

The problem with access policies becoming large and difficult to
reason about has also been studied, but not in the general case:
\cite{heckman98applying}~speaks of verifying a hierarchy of security
servers to ensure that they are enforcing the whole access policy,
and \cite{naumovich:permission-analysis}~presents an algorithm for
computing the flow of permissions through the Java security model, to
aid static analysis.  Neither of these are exactly what we want:
\cite{heckman98applying}~can prove that the programs you have
collectively implement the policy you specified, but their technique
cannot tell you whether you have made a subtle error in creating your
policy in the first place; \cite{naumovich:permission-analysis} is
more comprehensive but is specific to Java's security model.

Automated analysis of access control policies has also been
researched; \cite{schaad:lightweight}~and \cite{zao:rbac}~analyze role
based access control schemas using the Alloy modeling system which we
also use.  However \cite{schaad:lightweight}~uses Alloy to verify that
the composition of specifications is well formed and is silent about
their content, whereas we introduce a formal model of and a partial
ordering on XACML specifications specifically designed for analyzing
the semantics.  \cite{zao:rbac}~models RBAC schema in Alloy and then
checks these models against predicates, also written in Alloy.  We
introduce a formal model for XACML with a partial ordering on policies
that we then automatically check using Alloy as a back end; we do not
insist that the user write Alloy predicates directly and operate
solely on XACML.

{\bf XACML:}
XACML is an OASIS standard for specifying access policies; it is
written in XML.  The language comprises three classes of
objects---individual rules, collections of rules called policies, and
collections of policies called policy sets.  An XACML Policy
Enforcement Point, the gateway that determines whether an action is
permitted or not, takes \emph{access requests}, which are specially
formatted XML documents that define a set of data that we call the
\emph{environment}.  Policy Enforcement Points yield one of four
results: Permit, meaning that the access request is permitted; Deny,
meaning that the access request will not be permitted; Not Applicable,
meaning that this particular policy says nothing about the request;
and Indeterminate, which means that something unexpected came up and
the policy has failed.  Which result is yielded depends on what result
the policy dictates, given the environment defined in the access
request.

XACML rules are the most basic object, and have a goal effect---either
Permit or Deny---a domain of applicability, and conditions under which
they can yield Indeterminate and fail.  The domain of applicability is
realized in a series of predicates about the environmental data that
must all be satisfied for the rule to yield its goal effect; the error
conditions are embedded in the domain predicates, but can be separated
out into a set of predicates all their own.  Policies combine
individual rules and also have a domain of applicability; policy sets
combine individual policies with a domain of applicability.

XACML predicates comprise one of a number of primitive functions, with
mechanisms for extension; we consider only the core functionality.
These functions include simple equality, set inclusion, ordering
within numeric types, and also more complex functions such as XPath
matching and X500 name matching.

% arch-tag: 964a775e-9547-4b64-a0f7-ebf78d282c99

%%% Local Variables: 
%%% mode: latex
%%% TeX-master: "paper"
%%% End: 

\section{Policy Specifications}
\label{sec:policy-spec}

XACML is an OASIS standard for specifying access policies; it is
written in XML.  The language comprises three classes of
objects---individual rules, collections of rules called policies, and
collections of policies called policy sets.  An XACML Policy
Enforcement Point, the gateway that determines whether an action is
permitted or not, takes \emph{access requests}, which are specially
formatted XML documents that define a set of data that we call the
\emph{environment}.  Policy Enforcement Points yield one of four
results: Permit, meaning that the access request is permitted; Deny,
meaning that the access request will not be permitted; Not Applicable,
meaning that this particular policy says nothing about the request;
and Indeterminate, which means that something unexpected came up and
the policy has failed.  Which result is yielded depends on what result
the policy dictates, given the environment defined in the access
request.

XACML rules are the most basic object, and have a goal effect---either
Permit or Deny---a domain of applicability, and conditions under which
they can yield Indeterminate and fail.  The domain of applicability is
realized in a series of predicates about the environmental data that
must all be satisfied for the rule to yield its goal effect; the error
conditions are embedded in the domain predicates, but can be separated
out into a set of predicates all their own.  Policies combine
individual rules and also have a domain of applicability; policy sets
combine individual policies with a domain of applicability.

%% NOTE: Here we need to clarify what is the core functionality 
% that we are handling
XACML predicates comprise one of a number of primitive functions, with
mechanisms for extension; we consider only the core functionality.
These functions include simple equality, set inclusion, ordering
within numeric types, and also more complex functions such as XPath
matching and X500 name matching.

\begin{figure}[t]
\centering
\begin{scriptsize}
 \begin{verbatim}
<?xml version="1.0" encoding="UTF-8"?>
<Policy
  xmlns="urn:..."
  xmlns:xsi="...-instance"
  xmlns:md="http:.../record.xsd"
  PolicySetId="urn:example:policyid:1"
  RuleCombiningAlgId="urn:...:deny-overrides">
  <Target>
    <Subjects><AnySubject/></Subjects>
    <Resources><AnyResource/></Resources>
    <Actions>
      <Action>
        <ActionMatch MatchId="urn:...:string-equal">
          <AttributeValue DataType="...#string">
            vote
          </AttributeValue>
          <ActionAttributeDesignator
            AttributeId="urn:example:action"
            DataType="...#string"/>
        </ActionMatch>
      </Action>
    </Actions>
  </Target>
  <Rule RuleId="urn:example:ruleid:1" Effect="Deny">
    <Condition FunctionId="urn:...:integer-less-than">
      <Apply FunctionId="urn:...:integer-one-and-only">
        <SubjectAttributeDesignator
          AttributeId="urn:example:age"
          DataType="...#integer"/>
      </Apply>
      <AttributeValue DataType="...#integer">
        18
      </AttributeValue>
    </Condition>
  </Rule>
  <Rule RuleId="urn:example:ruleid:2" Effect="Deny">
    <Condition FunctionId="urn:...:boolean-equal">
      <Apply FunctionId="urn:...:boolean-one-and-only">
        <SubjectAttributeDesignator
          AttributeId="urn:example:voted-yet"
          DataType="...#boolean"/>
      </Apply>
      <AttributeValue DataType="...#boolean">
        True
      </AttributeValue>
    </Condition>
  </Rule>
  <Rule RuleId="urn:example:ruleid:3" Effect="Permit"/>
</Policy>
\end{verbatim}

\end{scriptsize}
\caption{A simple XACML policy}
\label{fig:example}
\end{figure}

Let us consider a simple example policy. 
The policy states that to be able to vote a person must be at least 18 years old
and a person who has voted already cannot vote. 
Our environment, the set of
information we are interested in, consists of the age of the person in
question and whether they have voted already.  We can represent this
as a Cartesian product of XML Schema~\cite{xmlschema} basic types,
as follows:
\begin{equation*}
  E = \mathcal{P} (\mtt{xsd:int}) \times \mathcal{P} (\mtt{xsd:boolean})
  \times \mathcal{P} (\mtt{xsd:string})
  \label{eqn:env}
\end{equation*}
The first component of the environment $E$ is the age of the
person, the second component is whether or not they have voted
already, and the third component is the action they are attempting
(perhaps voting, but perhaps something else).  We use power sets here
because in XACML all attributes describe sets of values, never
singletons.

The XACML policy for this
example is shown in Figure \ref{fig:example}.
We will explain the semantics of this policy
using a simple mathematical notation below. 
The goal for our policy is that if a person is doing something other than
voting, we do not really care what happens, and we require that there
be only one age and one voting record presented.  To do this we can
divide $E$ into four sets, $E_a$, $E_v$, $E_p$ and $E_d$ as follows
(note that the notation $\exists ! \, x \, P$ asserts that there is a
unique $x$ that satisfies a condition $P$):
\begin{eqnarray*}
  E_a & = & \{ \langle a, v, o \rangle \in E : \exists ! \, a_0 \in a 
  \wedge \exists ! \, v_0 \in v \} \\
  E_v & = & \{ \langle a, v, o \rangle \in E_a : \exists x \in o \; 
  x = \texttt{vote} \} \\
  E_p & = & \{ \langle \{a_0\}, \{v_0\}, o \rangle \in E_v : a_0 \geq 18 
  \wedge \neg v_0 \} \\
  E_d & = & E_v - E_p 
   =  \{ \langle \{a_0\}, \{v_0\}, o \rangle \in E_v : a_0 < 18 \vee v_0 \}
\end{eqnarray*}
Here, $E_a$ is the set of all environments whose inputs are not
erroneous, $E_v$ is the set of all environments where voting is
attempted, $E_p$ is the set of all environments where the person can
vote (their attempt to vote is \emph{permitted}), and $E_d$ is the set
of all environments where the person cannot vote (their attempt to
vote is \emph{denied}).  
In the following
section we will define a concise formal model for XACML policies.

\subsection{Formal Model}
\label{sec:formal-model}

Let $R = \{ \Permit, \Deny, \NotApplicable, \Indeterminate \}$ be the set
of valid results.  Now, we can define the set of valid policies $P$ as
follows (semantics will be defined later):
\begin{eqnarray*}
  \Permit & \in & P \\
  \Deny & \in & P \\
  \forall p \in P \;\, \forall S \subseteq E \; \Scope (p, S) & \in & P \\
  \forall p \in P \;\, \forall S \subseteq E \; \Err (p, S) & \in & P \\
  \forall p, q \in P \; p \oplus q & \in & P \\
  \forall p, q \in P \; p \ominus q & \in & P \\
  \forall p, q \in P \; p \otimes q & \in & P \\
  \forall p, q \in P \; p \oslash q & \in & P
\end{eqnarray*}
Informally, we regard \Permit{} and \Deny{} as symbols whose semantics
ignore the environment, always yielding \Permit{} or \Deny{},
respectively.  Along these same lines, \Scope{} and \Err{} attach
conditions to policies:
\begin{itemize}
\item $\Scope (p, S)$ modifies policy $p$ to yield $p$'s
answer if the current environment is in $S$, or \NotApplicable{}
otherwise.
\item $\Err (p, S)$ yields \Indeterminate{} if 
the current environment is in $S$ or $p$'s answer otherwise.  
\end{itemize}
The other four
symbols ($\oplus, \ominus, \otimes, \oslash$) are combinators, that
combine two policies in various ways:
\begin{itemize}
\item {\bf Permit-overrides:} $p \oplus q$ always yields
\Permit{} if either $p$ or $q$ yield \Permit{}.
\item  {\bf Deny-overrides:} $p \ominus q$ always
yields \Deny{} if either $p$ or $q$ yield \Deny{}.
\item  {\bf Only-one-applicable:} $p \otimes q$ requires that
one of $p$ or $q$ yield \NotApplicable{} and then yields the other
half's answer.
\item {\bf First-applicable:} $p \oslash q$ yields $p$'s answer unless that is
\NotApplicable{}, in which case it yields $q$'s answer.
\end{itemize}

More formally, we define a function $\eff : E \times P \rightarrow R$
that, given an environment and a policy produces a result.  We define
this function in Figure~\ref{fig:semantics} so that it corresponds to
our intuition about the desired properties, as described above.

% We define our semantics to be slightly different from that defined in
% the XACML standard; specifically, $\eff(e, \Permit \oplus
% \Indeterminate) = \Indeterminate$, not $\Permit$, and similarly
% $\eff(e, \Deny \ominus \Indeterminate) = \Indeterminate$ not $\Deny$.
% We did this because \Indeterminate{} is an erroneous condition in an
% XACML specification; we believed that for the purposes of analysis
% error conditions should not be covered up.  Using the original
% semantics does not change the analysis technique, merely some of the
% normal form transformations and then only slightly.

\begin{figure}[t]
\centering
\begin{footnotesize}
\begin{eqnarray*}
\eff : E \times P \rightarrow R  && \\
  \eff (e, \Permit) & = & \Permit \\
  \eff (e, \Deny) & = & \Deny \\
  \eff (e, \Scope (p, S)) & = & \left\{\begin{array}{ll}
  \eff (e, p) & \text{if $e \in S$} \\
  \NotApplicable & \text{otherwise}
  \end{array}\right. \\
  \eff (e, \Err (p, S)) & = & \left\{\begin{array}{ll}
  \Indeterminate & \text{if $e \in S$} \\
  \eff (e, p) & \text{otherwise}
  \end{array}\right. \\
  \eff (e, p \oplus q) & = & \left\{\begin{array}{ll}
  \Permit & \text{if $\eff(e, p) = \Permit$} \\
  & \text{$\quad \vee \eff(e, q) = \Permit$} \\
  \Indeterminate & \text{if } (\eff(e, p) = \Indeterminate \\
  & \quad \wedge \eff(e, q) \neq \Permit) \\
  &  \vee (\eff(e, q) = \Indeterminate \\
  & \quad \wedge \eff(e, p) \neq \Permit) \\
  \Deny & \text{if } (\eff(e, p) = \Deny \\
  & \quad \wedge \eff(e, q) \neq \Permit \\
  & \quad \wedge \eff(e, q) \neq \Indeterminate) \\
  & \vee (\eff(e, q) = \Deny \\
  & \quad \wedge \eff(e, p) \neq \Permit \\
  & \quad \wedge \eff(e, p) \neq \Indeterminate) \\
  \NotApplicable & \text{otherwise}
  \end{array}\right. \\
  \eff (e, p \ominus q) & = & \left\{\begin{array}{ll}
  \Deny & \text{if $\eff(e, p) = \Deny$} \\
  & \text{$\quad \vee (\eff(e, q) = \Deny$} \\
  \Indeterminate & \text{if } (\eff(e, p) = \Indeterminate \\
  & \quad \wedge \eff(e, q) \neq \Deny) \\
  & \vee (\eff(e, q) = \Indeterminate \\
  & \quad \wedge \eff(e, p) \neq \Deny) \\
  \Permit & \text{if } (\eff(e, p) = \Permit \\
  & \quad \wedge \eff(e, q) \neq \Deny \\
  & \quad \wedge \eff(e, q) \neq \Indeterminate) \\
  & \vee (\eff(e, q) = \Permit \\
  & \quad \wedge \eff(e, p) \neq \Deny \\
  & \quad \wedge \eff(e, p)\neq \Indeterminate) \\
  \NotApplicable & \text{otherwise}
  \end{array}\right. \\
  \eff (e, p \otimes q) & = & \left\{\begin{array}{ll}
  \eff (e, p) & \text{if $\eff(e, q) = \NotApplicable$} \\
  \eff (e, q) & \text{if $\eff(e, p) = \NotApplicable$} \\
  \Indeterminate & \text{otherwise}
  \end{array}\right. \\
  \eff (e, p \oslash q) & = & \left\{\begin{array}{ll}
  \eff (e, p) & \text{if $\eff(e, p) \neq \NotApplicable$} \\
  \eff (e, q) & \text{otherwise}
  \end{array}\right.
\end{eqnarray*}
\end{footnotesize}
\caption{Semantics of policies}
\label{fig:semantics}
\end{figure}

Using this notation, we can now model our example as follows:
\begin{eqnarray}
  S_0 & = & \{ \langle a, v, o \rangle \in E : \forall x \in a \; x < 18 \} 
  \label{eq:s0} \\
  S_1 & = & \{ \langle a, v, o \rangle \in E : \forall x \in v \; x \}
  \label{eq:s1} \\
  S_2 & = & \{ \langle a, v, o \rangle \in E : 
  \exists x \in o \; x = \mtt{vote} \} \label{eq:s2} \\
  S_3 & = & \{ \langle a, v, o \rangle \in E : \neg \exists ! \, a_0 \in a \} 
  \label{eq:s3} \\
  S_4 & = & \{ \langle a, v, o \rangle \in E : \neg \exists ! \, v_0 \in v \}
  \label{eq:s4} \\
  r_1 & = & \Err (\Scope (\Deny, S_0), S_3) \label{eq:r1} \\
  r_2 & = & \Err (\Scope (\Deny, S_1), S_4) \label{eq:r2} \\
  p & = & \Scope (r_1 \ominus r_2 \ominus \Permit, S_2) \label{eq:p}
\end{eqnarray}
Here, $S_0$ is the set of environments that fail the age requirement,
$S_1$ is the set of environments that fail the voting requirement,
$S_2$ is the set of environments where someone's trying to vote, etc.

\subsection{Policy Transformations}
\label{sec:transformations}

\begin{figure}[t]
\centering
\begin{footnotesize}
\[
\begin{array}{l}
  f : P   \rightarrow  P \nonumber \\
  f (\Scope (\Scope (X, S), R))  =  f (\Scope (X, R \cap S)) 
  \nonumber \\
  f (\Scope (\Err (X, S), R))  =  f (\Err (\Scope (X, R \setminus S), S \cap R)) 
  \nonumber \\
  f (\Scope (X \oplus Y, S))  =  \Scope (f (X), S) \oplus \Scope (f (Y), S) 
  \nonumber \\
  f (\Scope (X \ominus Y, S))  =  \Scope (f (X), S) \ominus \Scope (f (Y), S) 
  \nonumber \\
  f (\Scope (X \otimes Y, S))  =  \Scope (f (X), S) \otimes \Scope (f (Y), S)) 
  \nonumber \\
  f (\Scope (X \oslash Y, S))  =  \Scope (f (X), S) \oslash \Scope (f (Y), S) 
  \nonumber \\
  f (\Scope (P, S))  =  \Scope (f (P), S) 
     \ \text{if no other rules apply} \nonumber \\
  f (\Err (\Err (X, S), R))  =  f (\Err (X, R \cup S)) \nonumber \\
  f (\Err (\Scope (X, S), R))  =  f (\Err (\Scope (X, S \setminus R), R)) 
     \ \text{if $S \cup R \neq \emptyset$} \nonumber \\
  f (\Err (X \oplus Y, S))   =   \Err (f (X), S) \oplus \Err (f (Y), S) \nonumber \\
  f (\Err (X \ominus Y, S))  =   \Err (f (X), S) \ominus \Err (f (Y), S) \nonumber \\
  f (\Err (X \otimes Y, S))  =   \Err (f (X), S) \otimes \Err (f (Y), S) \nonumber \\
  f (\Err (X \oslash Y, S))  =   \Err (f (X), S) \oslash \Err (f (Y), S) \nonumber \\
  f (\Err (P, S))  =  \Err (f (P), S) 
     \  \text{if no other rules apply} \nonumber \\
  f (\Permit)  =  \Permit \nonumber \\
  f (\Deny)  =  \Deny \nonumber
\end{array}
\]
\end{footnotesize}
\caption{$\eff$-preserving transformations for reduction to normal form}
\label{fig:1streduction}
\end{figure}

We would like to perform analysis on this model, and it would be
easier to do this analysis if we could bring the model into a normal
form.  To do this, first we define equivalence:
\begin{equation*}
  P_1 \equiv P_2 \text{ iff } \forall e \in E \; \eff (e, P_1) = \eff (e, P_2)
\end{equation*}
We call a function $f$ that takes a policy and returns another
policy an \emph{$\eff$-preserving transformation} if $\forall p \in P
\; f (p) \equiv p$.

For any given policy, we want to regard the subset of $E$ that will
give a \Permit{} result, the subset of $E$ that will give a \Deny{}
result, and the subset of $E$ that will give an \Indeterminate{} result
independently.  We define the shorthand $\langle S, R, T \rangle$,
where $S$, $R$ and $T$ are pairwise disjoint, as follows:
\begin{equation*}
  \langle S, R, T \rangle = \Err (\Scope (\Permit, S) \otimes \Scope (\Deny, R), T)
\end{equation*}
Hence, $\langle S, R, T \rangle$ is simply a policy
that yields \Permit{} for any environment in $S$,
\Deny{} for any environment in $R$,
\Indeterminate{} for any environment in $T$,
and \NotApplicable{} for any remaining environment.
We call this \emph{triple notation} and refer to individual nodes
$\langle S, R, T \rangle$ as \emph{triples}.

Now that we have a framework for transforming policies, we would like
to transform an entire policy with $\Scope$, $\Err$ and combinators alike
into a single triple.  We know that for any policy $P$ a triple $P_T$
that is equivalent to it exists: the triple is just 
\[ 
\begin{array}{ll} 
P_T = & \langle \{
e \in E : \eff (e, P) = \Permit \}, \\
&  \{ e \in E : \eff (e, P) = \Deny \}, \\
& \{ e \in E : \eff (e, P) = \Indeterminate \}\rangle.
\end{array}
\]  
However,
this is not a constructive definition.  To transform the policies to
the triple form, we define two functions 
$f : P   \rightarrow  P$ and $g : P \rightarrow \langle S, R, T \rangle$, both
$\eff$-preserving transformations, such that $g (f(p))$ is a triple
for all $p$.  The $f$ function transforms the policy into an
equivalent one that is composed of triples joined by combinators.  The
$g$ function combines triples joined by combinators into a single
triple.  The two together generate the triple representation.  We
define $f$ in Figure~\ref{fig:1streduction}, and $g$ in
Figure~\ref{fig:triplereduction}.

As an example, applying $f$ to the policy $p$ defined in
Equation~\eqref{eq:p} leads to the following:
\begin{eqnarray*}
  p & = & \Scope (\Err (\Scope (\Deny, S_0), S_3) \\
  & & \qquad \ominus \Err (\Scope (\Deny, S_1), S_4) \\
  & & \qquad \ominus \Permit, S_2) \nonumber \\
  f(p)  & = & \Err (\Scope (\Deny, S_2 \cap S_0 \setminus S_3), S_3 \cap S_2) \\
  & & \qquad \ominus
  \Err (\Scope (\Deny, S_2 \cap S_1 \setminus S_4), S_4 \cap S_2) \\
  & & \qquad \ominus \Scope (\Permit, S_2)
\end{eqnarray*}
Note that the function $f$ pushes all $\Scope$ forms down to the leaves
of the policy tree, and all $\Err$ forms down to just above the leaves.

The $f$ function transforms a policy to a collection of expressions of
the form $\Err (\Scope (A, B), T)$ (where $A \in \{ \Permit, \Deny
\}$, $B, T \subseteq E$, and $B \cap T = \emptyset$) combined using
$\oplus, \ominus, \otimes$ and $\oslash$.  Since $\forall e \in E \,
\eff (e, X \otimes \Scope (Y, \emptyset)) = \eff (e, X)$, we can
further rewrite these expressions in the form $\Err (\Scope (\Permit,
S)$ $\otimes$ $\Scope (\Deny, R), T)$ combined with $\oplus, \ominus,
\otimes$ and $\oslash$ where $S = B$ and $R = \emptyset$ if $A =
\Permit$ and $S = \emptyset$ and $R = B$ if $A = \Deny$.  Since $S, R$
and $T$ are all pairwise disjoint this is exactly the required form
for our triple notation.  Hence, after applying the function $f$ we
have a set of subpolicies in our triple notation combined with
$\oplus, \ominus, \otimes$ and $\oslash$.  We define the function $g$
in Figure~\ref{fig:triplereduction}.  The transformations for function
$g$ all preserve the disjointness property, and using the function $g$
we can transform the policy generated by function $f$ to a single
triple $\langle S, R, T \rangle$ for some $S, R, T \subseteq E$.

\begin{figure}[t]
\centering
\begin{footnotesize}
\[
\begin{array}{l}
  g : P \rightarrow \langle S, R, T \rangle \\
  g (\langle S_1, R_1, T_1 \rangle \oplus \langle S_2, R_2, T_2 \rangle) = 
  \langle 
  S_1 \cup S_2, \\
  \qquad (R_1 \setminus (S_2 \cup T_2)) \cup (R_2 \setminus (S_1 \cup T_1)), 
  T_1 \cup T_2 \setminus (S_1 \cup S_2)
  \rangle \\
  g (\langle S_1, R_1, T_1 \rangle \ominus \langle S_2, R_2, T_2 \rangle)
  =  \\
  \qquad \langle (S_1 \setminus (R_2 \cup T_2)) \cup 
  (S_2 \setminus (R_1 \cup T_1)), \\
  \qquad
  R_1 \cup R_2,
  (T_1 \cup T_2) \setminus (R_1 \cup R_2)
  \rangle
  \\
  g (\langle S_1, R_1, T_1 \rangle \otimes \langle S_2, R_2, T_2 \rangle)
  = \\
  \qquad \langle 
  (S_1 \cup S_2) \setminus ((S_1 \cap S_2) \cup T_1 \cup T_2), \\
  \qquad
  (R_1 \cup R_2) \setminus ((R_1 \cap R_2) \cup T_1 \cup T_2), \\
  \qquad
  T_1 \cup T_2 \cup (S_1 \cap S_2) \cup (R_1 \cap R_2)
  \rangle 
  \\
  g (\langle S_1, R_1, T_1 \rangle \oslash \langle S_2, R_2, T_2 \rangle)
  = \langle 
  S_1 \cup (S_2 \setminus (R_1 \cup T_1)), \\
  \qquad 
  R_1 \cup (R_2 \setminus (S_1 \cup T_1)), 
  T_1 \cup (T_2 \setminus (S_1 \cup R_1))
  \rangle
  \\
  g (\langle S_1, R_1, T_1 \rangle) = \langle  S_1, R_1, T_1 \rangle 
  \\
  g (P_1 \oplus P_2) =  g (g (P_1) \oplus g (P_2)) \quad 
  \text{if no other rules apply}  \\
  g (P_1 \ominus P_2) =  g (g (P_1) \ominus g (P_2)) \quad 
  \text{if no other rules apply}  \\
  g (P_1 \otimes P_2) =  g (g (P_1) \otimes g (P_2)) \quad 
  \text{if no other rules apply}  \\
  g (P_1 \oslash P_2) =  g (g (P_1) \oslash g (P_2)) \quad
  \text{if no other rules apply} 
\end{array}
\]
\end{footnotesize}
\caption{$\eff$-preserving transformations for $\langle S, R, T \rangle$ 
  reduction}
\label{fig:triplereduction}
\end{figure}

When we apply the function $g$ to our example we get the following:
\begin{eqnarray*}
  f(p) & = & \Err (\Scope (\Deny, S_2 \cap S_0 \setminus S_3), S_3 \cap S_2) \\
  & & \ominus \Err (\Scope (\Deny, S_2 \cap S_1 \setminus S_4), S_4 \cap S_2) \\
  & & \ominus \Scope (\Permit, S_2)
  \\
  & = & \langle \emptyset, S_2 \cap S_0 \setminus S_3, S_3 \cap S_2 \rangle \\
  & & \ominus \langle \emptyset, S_2 \cap S_1 \setminus S_4, S_4 \cap S_2 \rangle \\
  & & \ominus \langle S_2, \emptyset, \emptyset \rangle
  \\
  g(f(p))  & = & \langle S_2 \setminus (S_0 \cup S_1 \cup S_3 \cup S_4), \\
  & & ((S_0 \cup S_1) \setminus (S_3 \cup S_4)) \cap S_2, \\
  & &  (S_3 \cup S_4) \cap S_2 \rangle
\end{eqnarray*}

Now that we have our policy into a form that is convenient for
analysis, we can begin to prove things about it.

% arch-tag: 834f4d24-c6fd-4bef-9e73-89b1efa5ca0b

%%% Local Variables: 
%%% mode: latex
%%% TeX-master: "paper"
%%% End: 

\section{Properties of Policies}
\label{sec:properties-policies}

In this section we will 
show that properties of policies can be 
expressed based on several partial ordering relations.
For example, we might want to prove
that a (possibly very complex) policy at least protects as much as
some simpler policy, and similarly we might want to guarantee that a
(possibly very complex) policy does not say anything outside of its
scope. Such properties can be expressed using the ordering relations
defined below.

Let $P_1 = \langle S_1, R_1, T_1 \rangle$ and let $P_2 =
\langle S_2, R_2, T_2 \rangle$ be two policies.  
We define the following
partial orders:
\begin{eqnarray*}
  P_1 \sqsubseteq_P P_2 & \mbox{iff} & S_1 \subseteq S_2 \\
  P_1 \sqsubseteq_D P_2 & \mbox{iff} & R_1 \subseteq R_2 \\
  P_1 \sqsubseteq_E P_2 & \mbox{iff} & T_1 \subseteq T_2 \\
  P_1 \sqsubseteq_{P,D,E} P_2 & \mbox{iff} & P_1 \sqsubseteq_P P_2
  \wedge P_1 \sqsubseteq_D P_2 \wedge P_1 \sqsubseteq_E P_2 
\end{eqnarray*}
Note that, we can define a partial order for
for any combination of of $P$, $D$ and $E$.
We define $P_1 \sqsubseteq P_2 \equiv P_1 \sqsubseteq_{P,D,E} P_2$.
We can regard $P_1 \sqsubseteq P_2$ as stating that for any $e \in E$
where $\eff (P_1, e) \neq \NotApplicable$, $\eff (P_2, e) = \eff (P_1,
e)$.

To demonstrate the use of these ordering relations, 
let us create a new policy; people are permitted
to check the current results of the election, for exit polls.  
We encode this with the
following policy 
\begin{eqnarray*}
  S_5 & = & \{ \langle a, v, o \rangle \in E : 
  \exists x \in o \; x = \mtt{getresult} \} \\
  r_3 & = & \Scope (\Err (\Permit, S_4), S_5)
\end{eqnarray*}
where $S_4$ is defined in Equation~\eqref{eq:s4}.
Now, we can create a composite policy as follows $p_c = p \oplus r_3$,
where $p$ is defined in Equation~\eqref{eq:p}.  This
policy has a bug---specifically, it permits people under 18 to vote in
certain circumstances---and we will demonstrate the usefulness of our
technique by showing this.  First, we perform our translations on this
new policy as above, getting:
\begin{eqnarray*}
  g(f(r_3)) & = & \langle S_5 \setminus S_4, \emptyset, S_4 \cap S_5 \rangle \\
  g(f(p_c)) & = & \langle ((S_2 \setminus (S_0 \cup S_1 \cup S_3 \cup S_4)) 
  \cup (S_5 \setminus S_4)), \\
  & & \;\, (((S_0 \setminus S_3) \cup (S_1 \setminus S_4)) \cap S_2) 
  \setminus (S_4 \cap S_5), \\
  & & \;\, ((S_4 \cap S_5) \cup ((S_3 \cup S_4) \setminus \\
  & & \;\, \qquad ((S_0 \setminus S_3) \cup 
  (S_1 \setminus S_4))) \cap S_2) \setminus \\
  & & \;\, \quad ((S_2 \setminus (S_0 \cup S_1 \cup S_3 \cup S_4)) 
  \cup (S_5 \setminus S_4)) \rangle \\
\end{eqnarray*}
where $S_0,S_1, S_3$ and $S_4$ are from
Equations~\eqref{eq:s1} to ~\eqref{eq:s4}.
Using set algebra we can simplify the expression for policy $p_c$ to
\begin{align*}
  g(f(p_c)) = \big\langle & ((S_2 \setminus (S_0 \cup S_1 \cup S_3)) 
  \cup S_5) \setminus S_4, \\
  & (((S_0 \setminus S_3) \cup (S_1 \setminus S_4)) \cap S_2) 
  \setminus (S_4 \cap S_5), \\
  & \big((S_4 \cap S_5) \setminus (S_2 \setminus (S_0 \cup S_1 \cup S_3))\big)
  \cup \\
  & \quad \big(((S_3 \cup S_4) \setminus ((S_0 \setminus S_3) \cup
  (S_1 \setminus S_4))) \cap S_2 \big) \big\rangle
\end{align*}
Now, we insist that this combined policy deny anyone trying to vote
who is under 18.  This is itself a policy, which we call $p_v$:
\begin{align*}
  p_v = \langle \emptyset, (S_0 \cap S_2) \setminus (S_3 \cup S_4),
  (S_3 \cup S_4) \cap S_2 \rangle
\end{align*}
The interesting thing here is whether or not $p_v \sqsubseteq_D p_c$,
i.e., does the policy $p_c$ deny every input that is denied by $p_v$.
That would mean that everyone trying to vote who is under 18 is
denied, and that our policy combination has not done any harm.
However, the environmental tuple \[e = \langle \{17\}, \{\mtt{true}\},
\{\mtt{vote}, \mtt{getresult} \} \rangle\] demonstrates that that is
not the case.  Input $e$ passes the second part of the \Permit{}
requirement and so is permitted by $p_c$ (which means that it is {\em
  not} denied by $p_c$) but denied by $p_v$, i.e., $e$ demonstrates
that $p_v \not \sqsubseteq_D p_c$.  The error is that, we do not
enforce that only one action be given in the third component of the
input, and because of this we have the surprising result that someone
who is under eighteen and has already voted, but asks for the voting
results at the same time as trying to vote will be permitted, and so
can cast any number of ballots.  To fix this, we could insist upon a
new condition, that $\exists ! \, x \in o$; or we could use $\otimes$
instead of $\oplus$, which would ensure that only one of the
sub-policies could be definitive on any given point (and so turn $\eff
(e, p_v)$ into an \Indeterminate{} result instead of a \Permit{}); or we
could decide that only people who have voted already can check the
results.

% arch-tag: e2b0e59e-49da-4b4c-9860-1da5c30b891f

%%% Local Variables: 
%%% mode: latex
%%% TeX-master: "paper"
%%% End: 


\def \Union {\bigcup}
\def \union {\cup}
\def \Intersect {\bigcap}
\def \intersect {\cap}
\def \And {\wedge}
\def \Or {\vee}
\def \band {\bigwedge}
\def \bor {\bigvee}
\def \Implies {\Rightarrow}
\def \Iff {\Leftrightarrow}
\def \implies {\rightarrow}
\def \iff {\leftrightarrow}
\def \true {\texttt{true}}
\def \false {\texttt{false}}
\def \assign {:=}

\section{Automatically Proving Properties of Policies}
\label{sec:autom-prov-prop}

Given the formal model defined in Section~\ref{sec:formal-model} and
properties defined in Section~\ref{sec:properties-policies} we would
like to check properties of access policies automatically.  To do this
we first formalize the syntax of formulas we use to specify subsets of
$E$.  Then we discuss how policies constructed using these formulas
and policy combinators can be translated to Boolean logic formulas.  
After this translation we show that we can 
check properties of access policies using a SAT solver.

\subsection{Characterizing Subsets of the Environment}
\label{sec:policy-classes}

In Section~\ref{sec:formal-model}, we defined our formal model using
subsets of the set of possible environments $E$. 
We showed that each policy can be expressed
in triple form $P= \langle S, R, T \rangle$   where
$S$, $R$, and $T$ are subsets $E$.
We will assume that all subsets of $E$ are specified in the form:
\[ 
\{ e \in E : C \}
\]
where $C$ is a constraint that evaluates to true or
false for each environment, i.e., the only free variables
in $C$ are the components of the environment tuple $e$.
Note that the sets $S_0, S_1, \dots, S_4$ in
Equation~\eqref{eq:s0} are expressed this way.

Given a set in the form $S = \{ e \in E : C \}$
our goal is to generate a boolean logic formula
$B$ which encodes the set $S$.  The encoding will map each $e \in E$
to a valuation of the boolean variables in $B$ and $B$ will evaluate
to true if and only if $e \in S$.  Based on such an encoding we can
convert questions about different policies (such as if one subsumes
the other one) to SAT problems and then use a SAT solver to check
them.  For example, we can generate a boolean formula which is
satisfiable if and only if an access policy is not subsumed (i.e.,
$\not \sqsubseteq$) by another one.  If the SAT solver returns a satisfying
assignment to the formula, then we can conclude that the property is
false, and generate a counterexample based on the satisfying
assignment.  If the SAT solver declares that the formula is not
satisfiable then we can conclude that the property holds.
We will discuss the details of such a translation below.

\begin{figure*}[t]
\[
\begin{array}{lll}
1 & SCAL \rightarrow  A &
        SCAL.f \assign SCAL.v[A]
        \And \band_{i=1, i \neq A}^k (\neg SCAL.v[i]) \\
2 & SCAL \rightarrow  a &
        SCAL.f \assign \band_{i=1}^k (BSET.v[i] \iff a[i])
        \And (\bor_{i=1}^k BSET.v[i]) \And \band_{i=1}^k
        (SCAL.v[i] \implies \band_{j=1,j\neq i}^k \neg SCAL.v[j])  \\
3 & BSET \rightarrow  s &
        BSET.f \assign \band_{i=1}^k (BSET.v[i] \iff s[i]) \\
4 & BSET \rightarrow  e[i] &
        SCAL.f \assign \band_{j=1}^k (BSET.v[j] \iff e[i][j]) \\
5 & SET  \rightarrow  \{ SCAL \} &
        SET.f \assign SCAL.f \And \band_{i=1}^k (SET.v[i] \iff SCAL.v[i]) \\
6 & SET  \rightarrow  BSET &
        SET.f \assign BSET.f \And \band_{i=1}^k (SET.v[i] \iff BSET.v[i]) \\
7 & SET  \rightarrow  SET_1 \cup SET_2 &
        SET.f \assign SET_1.f \And SET_2.f \And
        \band_{i=1}^k (SET.v[i] \iff (SET_1.v[i] \Or SET_2.v[i])) \\
8 & SET  \rightarrow SET_1 \cap SET_2 &
        SET.f \assign SET_1.f \And SET_2.f \And
        \band_{i=1}^k (SET.v[i] \iff (SET_1.v[i] \And SET_2.v[i])) \\
9 & SET  \rightarrow SET_1 \setminus SET_2 &
        SET.f \assign SET_1.f \And SET_2.f \And
        \band_{i=1}^k (SET.v[i] \iff (SET_1.v[i] \And \neg SET_2.v[i])) \\
10 & BP \rightarrow  \true &
        BP.f \assign BP.b \iff \true \\
11 & BP \rightarrow  \false &
        BP.f \assign BP.b \iff \false \\
12 & BP \rightarrow  SCAL_1 = SCAL_2 &
        BP.f \assign SCAL_1.f \And SCAL_2.f \And
        (BP.b \iff \band_{i=1}^k (SCAL_1.v[i] \iff SCAL_2.v[i])) \\
13 & BP \rightarrow  SCAL \in SET &
        BP.f \assign SCAL.f \And SET.f \And
        (BP.b \iff \band_{i=1}^k (SCAL.v[i] \implies SET.v[i])) \\
14 & BP \rightarrow  SET_1 \subseteq SET_2 &
        BP.f \assign SET_1.f \And SET_2.f \And
        (BP.b \iff \band_{i=1}^k (SET_1.v[i] \implies SET_2.v[i]))
\end{array}
\]
\caption{Translation of the basic predicates to Boolean logic formulas.}
\label{fig:BP}
\end{figure*}


To present our translation we use the following notational conveniences: for
elements $e \in E$, we name the components of $e$ $e[0]$, $e[1]$, $\dots$,
$e[n]$.  We use $s$, $s_0$, $s_1$, $\dots$, $s_n$ to denote set variables, 
$a$, $a_0$, $a_1$, $\dots$, $a_n$ to denote scalar variables, and $A$, $A_0$, $A_1$,
$\dots$, $A_n$ to denote constants.  Finally, $BP$ is a set of basic
predicates which we define as follows:
\begin{eqnarray*}
  SCAL & \rightarrow & A \quad | \quad a \\
  BSET & \rightarrow & s \quad | \quad e[i] \\
  BP & \rightarrow &  \texttt{true} \quad | \quad \texttt{false}  \\
 & | & SCAL = SCAL \quad | \quad SCAL \in SET  \\
 & | &  SET \subseteq SET \\
  SET & \rightarrow & BSET  \quad | \quad \{ SCAL \} \\
  & | & SET \cup SET \quad | \quad SET \cap SET  \\
  & | &  SET \setminus SET \\
\end{eqnarray*}
The above grammar is sufficient for specifying policies 
using only enumerated types (which
obviously have finite domains) and the simple operations $\neg$, $=$,
$\in$, $\subseteq$.
We will discuss extension to other domains later in this section.

Assuming that all subsets of $E$ are specified in the form $\{ e
\in E : C \}$, where there are no free variables save $e$ in $C$,
$C$ is defined as follows:
\begin{eqnarray*}
  C & \rightarrow & BP \quad | 
      \quad C \wedge C \quad | \quad C \vee C \quad | \quad \neg C \\
 & | &   \forall a \in BSET \; C \quad |
  \quad \exists a \in BSET \; C  \\
  & | & \exists ! \, a \in BSET \; C \quad 
\end{eqnarray*}
We use $\exists !$ to mean there exists exactly one instance that
holds.
We can express all set definitions on unordered and enumerated types
that are permitted in XACML using the expressions above.

We will explain our translation of a constraint $C$ defined by the
above grammar to a Boolean logic formula using attribute grammars.
We will first discuss the translation of the 
basic predicates $BP$.
In order to simplify our presentation we will assume that 
domains of all the scalars have the same size, call it $k$. We will encode a set
of values from any domain using a Boolean vector of size $k$.
Given a Boolean vector $v$, we will denote its components
as $v[1]$, $v[2]$, $\dots$, $v[k]$ where $v[i] \iff \true$ means
that element $i$ is a member of the set represented by $v$
whereas $v[i] \iff \false$ means that it is not.
We encode a set variable $s$ and each component of the
environment tuple $e$ using the same encoding, i.e., as a vector
of Boolean values.
To simplify our presentation we
also encode a scalar variable $a$ as a s set using a vector of Boolean values
but restrict it to be a singleton set by making sure that at any time
only one of the Boolean values in the vector can be true.
In our actual implementation scalar variables are represented
using $\log_2 k$ Boolean variables where $k$ is the size of the domain.

The attribute grammar for basic predicates is shown in 
Figure \ref{fig:BP}.
We numbered the production rules. Each production rule has a corresponding
semantic rule next to it. Semantic rules
describe how to compute the attributes of the nonterminal on the
left hand side of the production rule using the attributes
of the terminals and nonterminals on the right 
hand side of the production rule. 
In the attribute grammar shown in Figure \ref{fig:BP}
the nonterminals $SCAL$, $BSET$
and $SET$ have two attributes. One of them is a Boolean vector $v$
denoting a set of values, and the other one is a Boolean logic formula
$f$ which accumulates the frame constraints.
Again to simplify our presentation we represent scalar constants and
scalar variables (i.e., the non-terminal $SCAL$) as singleton sets
whereas in our actual implementation they are  represented
using $\log_2 k$ Boolean variables.

\begin{figure*}[t]
\[
\begin{array}{lll}
1 & C \rightarrow  BP &
        C.f \assign BP.f \And
        (C.b \iff BP.b) \\
2 & C \rightarrow  \neg C_1 &
        C.f \assign C_1.f \And
        (C.b \iff \neg C_1.b) \\
3 & C \rightarrow  C_1 \vee C_2 &
        C.f \assign C_1.f \And C_2.f \And
        (C.b \iff (C_1.b \Or C_2.b)) \\
4 & C \rightarrow  C_1 \wedge C_2 &
        C.f \assign C_1.f \And C_2.f \And
        (C.b \iff (C_1.b \And C_2.b)) \\
5 & C \rightarrow  \forall a \in BSET \; C_1  &
        C.f \assign BSET.f \And C_1.f \And
        \band_{i=1}^k (BSET.v[i] \implies
         (a[i] \And \band_{j=1,j \neq i}^k \neg a[j] \And C_1.b)) \\
6 & C \rightarrow  \exists a \in BSET \; C_1  &
        C.f \assign  BSET.f \And C_1.f \And
        (\bor_{i=1}^k (BSET.v[i] \implies
        (a[i] \And \band_{j=1,j \neq i}^k \neg a[j] \And C_1.b))) \\
7 & C \rightarrow  \exists ! \, a \in BSET \; C_1  &
  C.f \assign BSET.f \And C_1.f \And
  (\bor_{i=1}^k (BSET.v[i] \implies
   (a[i] \And \band_{j=1,j \neq i}^k \neg a[j] \And C_1.b))) \\
   & & \And (\bor_{i=1}^k ((BSET.v[i] \And
   a[i] \And \band_{j=1,j \neq i}^k \neg a[j] \And C_1.b) 
   \implies \neg \band_{l=1, l \neq i}^k (BSET.v[l] \And
   a[l] \And \band_{j=1,j \neq l}^k \neg a[j] \And C_1.b)))
\end{array}
\]
\caption{Translation of the constraints to Boolean logic formulas.}
\label{fig:C}
\end{figure*}

Rule 1 in Figure \ref{fig:BP} states that a scalar constant $A$
is encoded as a singleton set that contains only $A$. This singleton
set is represented as a Boolean vector $v$, such that 
$v[A]$ is set to true and all the rest of the elements of the vector
are set to false. This condition is stored in the frame constraint $f$.
Rule 2 states that a scalar variable is also encoded as a Boolean vector $v$.
The frame constraint $f$ makes sure that the elements of the Boolean vector $v$
are same as the elements of the Boolean vector representing the scalar
variable $a$ and exactly one of the elements in $a$ or $v$ is set to true
in any given time.
Rules 3 and 4 show that the set variables ($s$) and components of the
environment tuple ($e[i]$) are also encoded as Boolean vectors.

Rule 5 creates a singleton set from a scalar constant $SCAL$. However,
since we encode scalar constants as singleton sets, this simply means
that the Boolean vectors encoding the scalar constant ($SCAL.v$)
and the set ($SET.v$) are equivalent and the frame constraint
$SET.f$ expresses this constraint.
Note that in the attribute grammar shown in  Figure \ref{fig:BP}
the frame constraint of a nonterminal on the left hand side of a production
is a conjunction of the frame constraints of the nonterminals
on the right had side of the production plus some other constraints
that are added based on the production rule.

Rules 7, 8 and 9 encode the set operations: union, intersection and set
difference. Each set operation on two set expressions $SET_1$ and $SET_2$
results in the creation of new Boolean vector $SET.v$.
The value of an element in $SET.v$ is defined based on the corresponding
elements in  $SET_1.v$ and $SET_2.v$.
For example for the union operation
$SET.v[i]$ is true if and only if $SET_1.v[i]$ is true or $SET_2.v[i]$ is true.
The intersection and set difference are defined similarly.

The nonterminal $BP$ corresponds to the basic predicates and it has
two attributes. One of them is a boolean variable $b$ representing
the truth value of the predicate and the other one is a
Boolean logic formula $f$ that accumulates the frame constraints.

Rules 10 and 11 create two basic predicates which have the 
truth value true and false, respectively.

Rule 12 is a basic predicate that corresponds to an equality expression
comparing two scalars. Since scalars are expressed as Boolean vectors,
the Boolean variable encoding the truth value of the predicate  is
true if and only if all elements of the Boolean vectors encoding the two
scalar values are the same. This constraint is added to the frame constraint
of the basic predicate.

Rule 13 creates a basic predicate that corresponds to a membership
expression testing membership of a scalar to a set expression.
Rule 14 creates a basic predicate that corresponds to a subset
expression testing if a set expression is subsumed by another set expression.
Since we encode scalars a singleton sets the frame constraints  generated
for rules 13 and 14 are very similar. They just state that if a value
is a member of the set on the left hand side, then it should also be member of
the set on the right hand side.

The attribute grammar for the constraints is shown in
Figure \ref{fig:C}.
The nonterminals $C$ has two attributes.
One of them is a boolean variable $b$ representing
the truth value of the constraint and the other one is a
Boolean logic formula $f$ that accumulates the frame constraints.
Again, the frame constraint of a nonterminal 
on the left hand side of a production
is a conjunction of the frame constraints of the nonterminals
on the right had side of the production plus some other constraints
that are added based on the production rule.

Rule 1 is just a syntactic rule expressing that a constraint can be
a basic predicate.
Rule 2 defines the negation operation. As expected the
frame constraint states that the value of the constraint
on the left hand side of the production rule is the negation of
the the value of the constraint
on the right hand side of the production rule.
Rules 3 and 4 define the disjunction and conjunction operations.
The frame constraints generated in Rules 3 and 4
state that the value of the constraint
on the left hand side of the production rule is the disjunction or
the conjunction of the values of the constraints on the right
hand side of the production rule, respectively.

Rules 5, 6 and 7 deal with quantified constraints.
In the rules 5, 6 and 7, $a$ denotes a scalar variable which is quantified 
over a basic set expression $BSET$ which is either a set variable $s$ or
a component of the environment tuple $e[i]$.
The quantified variable $a$ can appear as a free variable in the constraint
expression on the right hand side ($C_1$).
Universal quantification is expressed as a conjunction which states that
for all the members of the set $s$ or $e[i]$, the constraint $C_1$ should evaluate
to true. This is achieved by restricting the value of the scalar variable $a$
to the value of a different member of the set for each conjunct.
Existential quantification is expressed similarly as a disjunction by 
restricting the value of the scalar variable $a$ to the value of a 
different member of the set for each disjunct.

Rule 7 is an existentially quantified constraint which evaluates to true
if and only if the constraint $C_1$ evaluates to true for exactly one
member of the set $s$ or $e[i]$. This is expressed by first stating
that there is at least one member of the set $s$ or $e[i]$ for which the
constraint $C_1$ evaluates to true (which is equivalent to
existential quantification) and then adding an extra conjunction which
states that the constraint $C_1$ does not evaluate to true for two
different members of the set $s$ or $e[i]$. 

The translation we described above can handle 
XACML policies which only use bounded unordered and enumerated types.
In fact, during our analysis we limit the size of every domain
to a given fixed size and then analyze the policies as though
they were specified using finite enumerated sets of that size. 
The problem is that if our automated analysis does not yield
a counterexample to a given property, then that does not necessarily mean that
no counterexample exists---perhaps if we had increased the scope just
a little more we would have found one. The small scope hypothesis
(discussed in~\cite{jackson:elements}, and tested and confirmed for
some data structure algorithms in~\cite{marinov:exhaustive}) suggests
that small scopes could be sufficient in practice. Note that if a
counterexample is found, that counterexample is definite and can be
translated into an error in the original policy.

Finally, the translation we described above does not handle domain specific
predicates, e.g., ordering relations on types such as integers.
When we translate sets described using such predicates to boolean logic
formulas we represent them as uninterpreted Boolean functions.
We create a Boolean
variable for encoding the value of the uninterpreted boolean function 
and we generate
constraints which guarantee that the value of the function is the same
if its arguments are the same. Other than this restriction the
variables encoding the functions can get arbitrary values.  Note that
this brings an extra level of imprecision to our analysis.  We were
not able to trust the positive results because of the scope
restriction, but now it is also possible that counterexamples may be
spurious, and will need to be validated against the original policy.
However, we think that such automated analysis can still be useful in
uncovering errors in access policies.

Note that, it is possible to fully interpret ordering relations.
We can encode a type with a domain of $n$
ordered elements using $n^2$ boolean variables, one for each pair of
values in the domain, representing the ordering relations.
However, XACML uses many complex functions such as XPath matching and X500
name matching which can be lead to very complex formulas if one tries to
fully interpret them in the Boolean logic translation.
Hence, we believe that using uninterpreted functions for abstracting such
complex functionality is a justified approach which enables us to handle
a significant portion of the XACML language.
Also, we would like to note that the imprecision caused by abstraction 
of such complex functions did not 
lead to any spurious results in the experiments we performed so far.

\subsection{Verification of Policies}

As discussed in Section \ref{sec:properties-policies}, we specify properties
of policies using a set of partial ordering relations. These partial ordering
relations can be used to state that a certain type of outcome
for one policy subsumes the same type of outcome for another policy.
In this section we will only focus on the $\sqsubseteq$ relation.
Translation of properties specified using other relations are handled similarly.

Given a query like $P_1 \sqsubseteq P_2$, our goal is to 
generate a Boolean logic formula which is satisfiable
if and only if $P_1 \not \sqsubseteq P_2$.
As we discussed earlier our tool first translates policies
$P_1$ and $P_2$ to triple form, such that
$P_1 = \langle S_1, R_1, T_1 \rangle$ and 
$P_2 = \langle S_2, R_2, T_2 \rangle$
where each element of each triple is specified with a constraint expression
as follows:
\begin{eqnarray*}
S_1  & = & \{ e \in E : C_{S_1} \} \\
R_1  & = & \{ e \in E : C_{R_1} \} \\
T_1  & = & \{ e \in E : C_{T_1} \} \\
S_2  & = & \{ e \in E : C_{S_2} \} \\
R_2  & = & \{ e \in E : C_{R_2} \} \\
T_2  & = & \{ e \in E : C_{T_2} \}
\end{eqnarray*}

After translating policies $P_1$ and $P_2$ in to the triple form
our translator generates a boolean logic formulas for the constraints
$C_{S_1}$,  $C_{R_1}$,  $C_{T_1}$, $C_{S_2}$, $C_{R_2}$ and $C_{T_2}$
based on the attribute grammar rules described in Figures \ref{fig:BP} and
\ref{fig:C}.
For example, after this translation the truth value of the
constraint $C_{S_1}$ is represented with the Boolean variable
$C_{S_1}.b$ and the frame constraint $C_{S_1}.f$ states all the
constraints on the Boolean variable $C_{S_1}.b$.

Recall that, given $P_1 = \langle S_1, R_1, T_1 \rangle$ and $P_2 =
\langle S_2, R_2, T_2 \rangle$, 
$P_1 \sqsubseteq P_2$ holds if and only if
\[
  S_1 \subseteq S_2
  \wedge R_1 \subseteq R_2 \wedge T_1 \subseteq T_2
\]
Based on this, we generate a formula $F$ such that 
\[
F = \true \ \text{ iff } \ P_1 \sqsubseteq P_2
\]
as follows:
\begin{eqnarray*}
F & = & (C_{S_1}.f \wedge C_{R_1}.f \wedge C_{T_1}.f \wedge 
    C_{S_2}.f \wedge C_{R_2}.f \wedge C_{T_2}.f ) \implies \\
  & &   ((C_{S_1}.b \implies C_{S_2}.b) \wedge (C_{R_1}.b \implies C_{R_2}.b)
    \wedge (C_{T_1}.b \implies C_{T_2}.b))
\end{eqnarray*}

Finally, we send the property $\neg F$ to the SAT solver. If the SAT solver
returns a satisfying assignment for the Boolean variables
encoding the environment tuple $e$ (which are the only free variables in the
formula $\neg F$), the satisfying assignment corresponds to 
a counter-example environment demonstrating how the property is violated. If the
SAT solver states that $\neg F$  is not satisfiable, then we conclude that
the property holds, i.e., $P_1 \sqsubseteq P_2$.

Since majority of the
SAT solvers expect their input to
be expressed in Conjunctive Normal Form (CNF), 
the last step in our translation (before we send the formula $\neg F$ 
to the SAT solver) is to convert $\neg F$ to CNF.
For conversion to CNF we implemented the structure
preserving technique from \cite{plaisted86structure}.

% arch-tag: 931ea1c1-9ebf-4e94-89e7-06ac1ad3856f

%%% Local Variables: 
%%% mode: latex
%%% TeX-master: "paper"
%%% End: 

\section{Experiments}
\label{sec:experiments}

Our tool generates Alloy code which is then run through the Alloy
Analyzer to do the analysis.  It is targeted for proving things about
$\sqsubseteq$ relations, but we can use it for simpler questions as
well.

One such question might be `give a tuple $e$ such that $\eff (e, p) =
\Permit$' (where $p$ is defined as in our running example).  We
generate the Alloy code for the XACML policy, as normal, and then
append the following:
\begin{verbatim}
fun CheckTuple {
    some T0.permit
}
run CheckTuple for 2 but 2 Bool, 1 Triple, 8 Type
\end{verbatim}
The numbers after \texttt{CheckTuple} establish how deeply we will
look; we're looking for such a tuple in a universe where there are two
Boolean values, only one Triple (\texttt{T0}, the one generated
through $\translation(p)$), 8 domain types (strings, integers, and the
like)), and two elements of everything else.  If we run this, the
analyzer tells us that the tuple $\langle \{ \mtt{Type\_6} \},$ $\{
\mtt{Bool\_1} \},$ $\{ \mtt{Type\_7} \} \rangle$ generates a
\Permit{}, and further examination of the output shows that
\texttt{Bool\_1} is \texttt{True}, \texttt{Type\_6} is 18, and
\texttt{Type\_7} is the string \texttt{vote}.  Through similar means
we can discover that the tuple $\langle \{ 18 \}, \{ \}, \{ \mtt{vote}
\} \rangle$ will generate an error.

For something more interesting, we try to show that $p_v \sqsubseteq_D
p_c$, as we proved manually.  Our coda now becomes
\begin{verbatim}
assert Subset {
    T0.deny in T1.deny
}
check Subset for 2 but 2 Bool, 2 Triple, 10 Type
\end{verbatim}
We get a counterexample almost immediately, giving us the tuple we
constructed in Section~\ref{sec:properties-policies}.  If we modify
the policy so as to restrict result checking to only those who voted
successfully, then the subset relation holds, and no counterexample is
produced.

\subsection{Timing Data}

These are but small examples, so they do not reflect the time required
to solve large problems that one might reasonably ask.  Since the
underlying problem (SAT) is NP complete, it is reasonable to ask
whether these techniques are useful at all as the problem size
increases.

There are two sides to this.  One side is that larger and more complex
policies will inescapably take longer than small policies.  The other
is that---in the context of problems in $BP_3$ and $BP_4$, where we
must restrain the domains to a certain finite size---the amount of
computation involved as the size of the domain increases may trigger
the exponential worst case behavior.

\begin{figure}
  \centering
  \includegraphics{data/chart}
  \caption{Run time vs. domain size plotted for $P_1$ and $P_2$}
  \label{fig:graph}
\end{figure}

\begin{table}
\centering
\begin{tabular}{ccc}
Domain size & \multicolumn{2}{c}{Comparison} \\
& $P_3 \sqsubseteq P_4$ & $P_4 \not\sqsubseteq P_3$ \\
\input{data/invloutput}
\end{tabular}
\caption{Median run time of Medico example ($P_4$) and subset ($P_3$) in seconds}
\label{tbl:bigdata}
\end{table}

To demonstrate that the analysis is still feasible, we have collected
two sets of data.  The first, which we have charted in
Figure~\ref{fig:graph}, shows the time required to verify or refute a
relationship between two example policies.  It shows how the technique
scales as the size of the domain sets increases.  The second set of
data proves and refutes a relationship between the Medico policy from
Section~4.2 of the XACML specification~\cite{xacml} and a subset of
itself.  This shows that the technique is feasible for larger
policies.  All these benchmarks were performed on a 1 GHz PowerPC, and
all times are the median of five runs, to smooth out irregularities.
In each case, trials were run until the formula proved too large for
Alloy Analyzer to handle; past the given sizes the analyzer would fail
cryptically.  ``Domain size'' means the number of elements we are
using for our analysis, in each domain; so a domain size of 8 for our
example environment $E$ means we simulate every $e \in E$ where
$|e[i]| \leq 8$ for each component in $E$.

The data indicates that the time required for analysis is exponential
in the size of the scope, which is to be expected for a SAT based
algorithm.  However, all times are under two minutes, and our
technique can clearly prove important properties of these problems in
a useful amount of time.

% arch-tag: 5a3a4d97-641f-4c94-804d-9ec83256e4e6

%%% Local Variables: 
%%% mode: latex
%%% TeX-master: "paper"
%%% End: 

\section{Conclusion}

We have presented a formal model for access policies, and shown how to
analyze interesting properties about such models in an automated way.
We have implemented a tool to translate XACML policies into this model
and to yield code suitable for analysis.  The experimental results
indicate that automated analysis of nontrivial access policies is
feasible.

It would be interesting to investigate using predicate abstraction to
generate more precise models for the functions that we cannot directly
simulate.  We would also like to experiment on more and larger
policies.

% arch-tag: d0400f29-ab24-4a88-a786-3e22d798dc41

%%% Local Variables: 
%%% mode: latex
%%% TeX-master: "paper"
%%% End: 


\bibliographystyle{abbrv}
%\bibliography{citations/citations.bib}
\begin{thebibliography}{10}

\bibitem{abadpeiro99plas}
J.~L. Abad-Peiro, H.~Debar, T.~Schweinberger, and P.~Trommler.
\newblock {PLAS} --- {P}olicy language for authorizations.
\newblock Technical Report RZ 3126, IBM Research Division, 1999.

\bibitem{abadi93calculus}
M.~Abadi, M.~Burrows, B.~Lampson, and G.~Plotkin.
\newblock A calculus for access control in distributed systems.
\newblock {\em ACM Transactions on Programming Languages and Systems},
  15(4):706--734, September 1993.

\bibitem{bertino98access}
E.~Bertino, C.~Bettini, E.~Ferrari, and P.~Samarati.
\newblock An access control model supporting periodicity constraints and
  temporal reasoning.
\newblock {\em ACM Transactions on Database Systems}, 23(3):231--285, 1998.

\bibitem{bonatti01dataarchives}
P.~Bonatti, E.~Damiani, S.~{De Capitani di Vimercati}, and P.~Samarati.
\newblock An access control model for data archives.
\newblock In {\em Proceedings of the 16th international conference on
  information security: trusted information}, pages 261--276, Paris, France,
  2001. Kluwer International Federation For Information Processing Series.

\bibitem{bonatti02algebra}
P.~Bonatti, S.~{De Capitani di Vimercati}, and P.~Samarati.
\newblock An algebra for composing access control policies.
\newblock {\em ACM Transactions on Information Systems Security}, 5(1):1--35,
  2002.

\bibitem{damiani02SVG}
E.~Damiani, S.~{De Capitani di Vimercati}, E.~Fern\'andez-Medina, and
  P.~Samarati.
\newblock Access control of {SVG} documents.
\newblock In {\em Proceedings of DBSec 2002}, pages 219--230, 2002.

\bibitem{damiani00design}
E.~Damiani, S.~{De Capitani di Vimercati}, S.~Paraboschi, and P.~Samarati.
\newblock Design and implementation of an access control processor for {XML}
  documents.
\newblock {\em Computer Networks}, 33(1--6):59--75, 2000.

\bibitem{damiani02fine}
E.~Damiani, S.~{De Capitani di Vimercati}, S.~Paraboschi, and P.~Samarati.
\newblock A fine-grained access control system for {XML} documents.
\newblock {\em ACM Transactions on Information Systems Security},
  5(2):169--202, 2002.

\bibitem{damiani01controllingaccess}
E.~Damiani, P.~Samarati, S.~{De Capitani di Vimercati}, and S.~Paraboschi.
\newblock Controlling access to {XML} documents.
\newblock {\em IEEE Internet Computing}, 5(6):18--28, 2001.

\bibitem{divimercati96federated}
S.~{De Capitani di Vimercati} and P.~Samarati.
\newblock Access control in federated systems.
\newblock In {\em Proceedings of the 1996 workshop on new security paradigms},
  pages 87--99, Lake Arrowhead, California, United States, 1996. ACM Press.

\bibitem{heckman98applying}
M.~Heckman and K.~N. Levitt.
\newblock Applying the composition principle to verify a hierarchy of security
  servers.
\newblock In {\em {HICSS} (3)}, pages 338--347, 1998.

\bibitem{jackson:micromodels}
D.~Jackson.
\newblock Micromodels of software: Modeling and analysis with {A}lloy.
\newblock \texttt{http://sdg.lcs.mit.edu/alloy/reference-manual.pdf}.

\bibitem{jackson00automating}
D.~Jackson.
\newblock Automating first-order relational logic.
\newblock In {\em Proc. ACM SIGSOFT COnf. Foundations of Software Engineering},
  November 2000.

\bibitem{jackson:elements}
D.~Jackson and C.~A. Damon.
\newblock Elements of style: Analyzing a software design feature with a
  counterexample detector.
\newblock {\em IEEE Transactions on Software Engineering}, 22(7), July 1996.

\bibitem{jackson:alcoa}
D.~Jackson, I.~Schechter, and I.~Shlyakhter.
\newblock Alcoa: the {A}lloy constraint analyzer.
\newblock In {\em Proceedings of International Conference on Software
  Engineering}, Limerick, Ireland, June 2000. IEEE.

\bibitem{jackson01lightweight}
D.~Jackson and A.~Waingold.
\newblock Lightweight extraction of object models from bytecode.
\newblock In {\em IEEE Transactions on Software Engineering}, February 2001.

\bibitem{jajodia01multiplepolicies}
S.~Jajodia, P.~Samarati, M.~L. Sapino, and V.~S. Subrahmanian.
\newblock Flexible support for multiple access control policies.
\newblock {\em ACM Transactions on Database Systems}, 26(2):214--260, 2001.

\bibitem{jajodialogical}
S.~Jajodia, P.~Samarati, and V.~S. Subrahmanian.
\newblock A logical language for expressing authorizations.
\newblock In {\em Proceedings of the 1997 IEEE Symposium on Security and
  Privacy}, pages 31--42, Oakland, CA, USA, 1997. IEEE Press.

\bibitem{jajoida97unified}
S.~Jajodia, P.~Samarati, V.~S. Subrahmanian, and E.~Bertino.
\newblock A unified framework for enforcing multiple access control policies.
\newblock In {\em SIGMOD'97}, pages 474--485, Tucson, AZ, May 1997.

\bibitem{marinov:exhaustive}
D.~Marinov, A.~Andoni, D.~Danilinc, S.~Khurshid, and M.~Rinard.
\newblock An evaluation of exhaustive testing for data structures.
\newblock Technical Report MIT-LCS-TR-921, MIT CSAIL, 2003.

\bibitem{marinov02valloy}
D.~Marinov and S.~Khurshid.
\newblock {VA}lloy: Virtual functions meet a relational language.
\newblock In {\em 11th International Symposium of Formal Methods Europe (FME
  2002)}, Copenhagen, Denmark, July 2002.

\bibitem{naumovich:permission-analysis}
G.~Naumovich.
\newblock A conservative algorithm for computing the flow of permissions in
  {J}ava programs.
\newblock In {\em Proceedings of the International Symposium on Software
  Testing and Analysis ({ISSTA} '02)}, pages 33--43, July 2002.

\bibitem{nolte02filesystem}
T.~Nolte.
\newblock Exploring filesystem synchronization with lightweight modeling and
  analysis.
\newblock Master's thesis, MIT, August 2002.

\bibitem{samarati01policiesmodelsmechanisms}
P.~Samarati and S.~{De Capitani di Vimercati}.
\newblock {\em Foundations of Security Analysis and Design}, chapter~3, pages
  137--196.
\newblock Springer Verlag, 2001.

\bibitem{sandhu96audit}
R.~Sandhu and P.~Samarati.
\newblock Authentication, access control, and audit.
\newblock {\em ACM Computing Surveys}, 28(1):241--243, 1996.

\bibitem{sandhu93latticebased}
R.~S. Sandhu.
\newblock Lattice-based access control models.
\newblock {\em IEEE Computer}, 26(11):9--19, November 1993.

\balancecolumns
\bibitem{sandhu96rolebased}
R.~S. Sandhu, E.~J. Coyne, H.~L. Feinstein, and C.~E. Youman.
\newblock Role-based access control models.
\newblock {\em IEEE Computer}, 29(2):38--47, 1996.

\bibitem{sandhu94access}
R.~S. Sandhu and P.~Samarati.
\newblock Access control: Principles and practice.
\newblock {\em IEEE Communications Magazine}, 32(9):40--48, 1994 1994.

\bibitem{schaad:lightweight}
A.~Schaad and J.~Moffet.
\newblock A lightweight approach to specification and analysis of role-based access control extensions.
\newblock In {\em 7th ACM Symposium on Access Control Models and Technologies (SACMAT 2002)}, June 2002.

\bibitem{vaziri03heap}
M.~Vaziri and D.~Jackson.
\newblock Checking heap-manipulating procedures with a constraint solver.
\newblock In {\em TACAS'03}, Warsaw, Poland, 2003.

\bibitem{waingold01automated}
A.~Waingold.
\newblock Automated extraction of abstract object models.
\newblock Master's thesis, MIT, May 2001.

\bibitem{zao:rbac}
J.~Zao, H.~Wee, J.~Chu, and D.~Jackson.
\newblock {RBAC} schema verification using lightweight formal model and constraint analysis.
\newblock XXX

\bibitem{xacml}
{eXtensible Access Control Markup Language (XACML)} version 1.0.
\newblock OASIS Standard, February 2003.

\bibitem{XML}
Extensible markup language ({XML}) 1.0 (second edition), 2000.

\bibitem{xmlschema}
{XML Schema} part 2: Datatypes.
\newblock W3C Recommendation, May 2001.

\end{thebibliography}

\clearpage
\appendix
\section{XACML Representation of Voting Example and Alloy Translation}
\label{sec:xacml-repr-voting}

{ \small
\begin{verbatim}
<?xml version="1.0" encoding="UTF-8"?>
<Policy
  xmlns="urn:..."
  xmlns:xsi="...-instance"
  xmlns:md="http:.../record.xsd"
  PolicySetId="urn:example:policyid:1"
  RuleCombiningAlgId="urn:...:deny-overrides">
  <Target>
    <Subjects><AnySubject/></Subjects>
    <Resources><AnyResource/></Resources>
    <Actions>
      <Action>
        <ActionMatch MatchId="urn:...:string-equal">
          <AttributeValue DataType="...#string">
            vote
          </AttributeValue>
          <ActionAttributeDesignator
            AttributeId="urn:example:action"
            DataType="...#string"/>
        </ActionMatch>
      </Action>
    </Actions>
  </Target>
  <Rule RuleId="urn:example:ruleid:1" Effect="Deny">
    <Condition FunctionId="urn:...:integer-less-than">
      <Apply FunctionId="urn:...:integer-one-and-only">
        <SubjectAttributeDesignator
          AttributeId="urn:example:age"
          DataType="...#integer"/>
      </Apply>
      <AttributeValue DataType="...#integer">
        18
      </AttributeValue>
    </Condition>
  </Rule>
  <Rule RuleId="urn:example:ruleid:2" Effect="Deny">
    <Condition FunctionId="urn:...:boolean-equal">
      <Apply FunctionId="urn:...:boolean-one-and-only">
        <SubjectAttributeDesignator
          AttributeId="urn:example:voted-yet"
          DataType="...#boolean"/>
      </Apply>
      <AttributeValue DataType="...#boolean">
        True
      </AttributeValue>
    </Condition>
  </Rule>
  <Rule RuleId="urn:example:ruleid:3" Effect="Permit"/>
</Policy>
\end{verbatim}

}
\newpage

\section{Alloy Translation}
\label{sec:alloy-trans}

{ \small
\input{trimmedexample.als.tex}
}

\end{document}
