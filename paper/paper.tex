\documentclass{acm_proc_article-sp}
\usepackage{amsmath}
\usepackage{times}
\usepackage{mathptm}
\renewcommand{\ttdefault}{cmtt}
\DeclareMathOperator{\eff}{eff}
\DeclareMathOperator{\xpathmatch}{xpathnodematch}
\DeclareMathOperator{\translation}{translate}
\DeclareMathOperator{\extract}{extract}
\DeclareMathOperator{\constant}{constant}
\DeclareMathOperator{\env}{env}
\newcommand{\step}[1]{\intertext{\{#1\}}}
\newcommand{\mtt}[1]{\text{\texttt{#1}}}
\newcommand{\itmathcommand}[1]{\ensuremath{\text{\textit{#1}}}}
\newcommand{\Scope}{\itmathcommand{Scope}}
\newcommand{\Err}{\itmathcommand{Err}}
\newcommand{\Permit}{\itmathcommand{Permit}}
\newcommand{\Deny}{\itmathcommand{Deny}}
\newcommand{\Indeterminate}{\itmathcommand{Indet}}
\newcommand{\NotApplicable}{\itmathcommand{NotApp}}

\begin{document}
\title{Automated Verification of Access Control Policies}
\numberofauthors{2}
\author{
\alignauthor Graham Hughes \quad Tevfik Bultan \\
      \affaddr{Computer Science Department} \\
      \affaddr{University of California} \\
      \affaddr{Santa Barbara, CA 93106, USA} \\
      \email{\{graham,bultan\}@cs.ucsb.edu} \\
}
\date\today
\maketitle

\begin{abstract}
  Managing access control policies in modern computer systems can be
  challenging and error-prone, especially when multiple access
  policies are combined to form new policies, possibly introducing
  unintended consequences.  In this paper we present a framework for
  automated verification of access control policies.  We introduce a
  formal model for systematically specifying access to resources.  We
  show that the access control policies in the XACML access control
  language can be translated to a simple form which partitions the
  input domain to four classes: permit, deny, error, and
  not-applicable.  We present several ordering relations for access
  control policies which can be used to specify the properties of
  the policies and the relationships among them.  We then show how to
  automatically check these ordering relations using an existing
  automated analysis tool.  In particular, we translate XACML policies
  to the Alloy language and check their properties using the Alloy
  Analyzer.  Our experimental results demonstrate that automated
  verification of XACML policies is feasible.
\end{abstract}


\section{Introduction}

Keeping track of permission grants, organizational policies and
special cases in a modern software system has become difficult enough
on its own; keeping the policy consistent across multiple
heterogeneous systems is even more difficult, as each system requires
that these grants and policies be expressed in its own specific
control language.  Several unified access policy languages address
this problem.  If all the systems use the same access policy language,
then access policies need only be written once and similarly only one
policy needs to be kept up to date.  In this paper we focus on one
particular such language, the OASIS standard XACML~\cite{xacml}.

OASIS (the Organization for the Advancement of Structured Information
Standards) is an international standards consortium noted particularly
for work in the popular markup language XML~\cite{XML}.  They have
published a standard for access policy languages called XACML
(expanded as ``eXtensible Access Control Markup Language''), which we
will discuss in detail below.  For the moment, it suffices to describe
it as an XML-based language for expressing access rights to arbitrary
objects that are identified in XML, with a particular focus on the
composition of many different subpolicies into a single superpolicy.

Having such a combined policy is convenient, but such a policy will
inevitably become quite large and complex as all an organization's
rules get placed in it.  It is possible, even likely, that the act of
creating a unified policy out of numerous disparate smaller policies
could leave it vulnerable to unintended consequences.  In this paper
we investigate statically verifying properties of access control
policies to prevent such errors. We translate XACML policies into a
simplified mathematical model, which we reduce to a normal form
separating the conditions that give rise to {\em access permitted},
{\em access denied}, and {\em internal error} results. We define
partial orderings between access control policies, with the intention
of checking whether a policy is over- or under constrained with
respect to another one.  We show that these ordering relations can be
translated to Boolean formulas which are satisfiable if and only
if the corresponding relation is violated.  We use a SAT solver to check
satisfiablity of these Boolean logic formulas. Using our
translator and a SAT solver we can check if a combination of XACML
policies does or does not faithfully reproduce the properties of its
subpolicies, and thus discover unintended consequences before they
appear in practice.

In Section~\ref{sec:policy-spec}, after giving an overview of XACML,
we develop a formal model for access policies; in
Section~\ref{sec:transformations} we discuss how to transform these
models into a normal form that distinguishes access permitted, access
denied, and error conditions.  In
Section~\ref{sec:properties-policies} we define partial ordering
relations among access policies which are used to specify their
properties.  We show how to check these properties automatically in
Section~\ref{sec:autom-prov-prop}.  Finally, we report the results of
our experiments in Section~\ref{sec:experiments}.


% arch-tag: 964a775e-9547-4b64-a0f7-ebf78d282c99

%%% Local Variables: 
%%% mode: latex
%%% TeX-master: "paper"
%%% End: 

\section{Policy Specifications}
\label{sec:policy-spec}

XACML is an OASIS standard for specifying access policies; it is
written in XML.  The language comprises three classes of
objects---individual rules, collections of rules called policies, and
collections of policies called policy sets.  An XACML Policy
Enforcement Point, the gateway that determines whether an action is
permitted or not, takes \emph{access requests}, which are specially
formatted XML documents that define a set of data that we call the
\emph{environment}.  Policy Enforcement Points yield one of four
results: Permit, meaning that the access request is permitted; Deny,
meaning that the access request will not be permitted; Not Applicable,
meaning that this particular policy says nothing about the request;
and Indeterminate, which means that something unexpected came up and
the policy has failed.  Which result is yielded depends on what result
the policy dictates, given the environment defined in the access
request.

XACML rules are the most basic object, and have a goal effect---either
Permit or Deny---a domain of applicability, and conditions under which
they can yield Indeterminate and fail.  The domain of applicability is
realized in a series of predicates about the environmental data that
must all be satisfied for the rule to yield its goal effect; the error
conditions are embedded in the domain predicates, but can be separated
out into a set of predicates all their own.  Policies combine
individual rules and also have a domain of applicability; policy sets
combine individual policies with a domain of applicability.

XACML predicates comprise one of a number of primitive functions, with
mechanisms for extension; we do not consider any extensions.  These
functions include simple equality, set inclusion, ordering within
numeric types, and also more complex functions such as XPath matching
and X500 name matching.

\begin{figure}[t]
\centering
\begin{scriptsize}
 \begin{verbatim}
<?xml version="1.0" encoding="UTF-8"?>
<Policy
  xmlns="urn:..."
  xmlns:xsi="...-instance"
  xmlns:md="http:.../record.xsd"
  PolicySetId="urn:example:policyid:1"
  RuleCombiningAlgId="urn:...:deny-overrides">
  <Target>
    <Subjects><AnySubject/></Subjects>
    <Resources><AnyResource/></Resources>
    <Actions>
      <Action>
        <ActionMatch MatchId="urn:...:string-equal">
          <AttributeValue DataType="...#string">
            vote
          </AttributeValue>
          <ActionAttributeDesignator
            AttributeId="urn:example:action"
            DataType="...#string"/>
        </ActionMatch>
      </Action>
    </Actions>
  </Target>
  <Rule RuleId="urn:example:ruleid:1" Effect="Deny">
    <Condition FunctionId="urn:...:integer-less-than">
      <Apply FunctionId="urn:...:integer-one-and-only">
        <SubjectAttributeDesignator
          AttributeId="urn:example:age"
          DataType="...#integer"/>
      </Apply>
      <AttributeValue DataType="...#integer">
        18
      </AttributeValue>
    </Condition>
  </Rule>
  <Rule RuleId="urn:example:ruleid:2" Effect="Deny">
    <Condition FunctionId="urn:...:boolean-equal">
      <Apply FunctionId="urn:...:boolean-one-and-only">
        <SubjectAttributeDesignator
          AttributeId="urn:example:voted-yet"
          DataType="...#boolean"/>
      </Apply>
      <AttributeValue DataType="...#boolean">
        True
      </AttributeValue>
    </Condition>
  </Rule>
  <Rule RuleId="urn:example:ruleid:3" Effect="Permit"/>
</Policy>
\end{verbatim}

\end{scriptsize}
\caption{A simple XACML policy}
\label{fig:example}
\end{figure}

Let us consider a simple example policy. 
The policy states that to be able to vote a person must be at least 18 years old
and a person who has voted already cannot vote. 
Our environment, the set of
information we are interested in, consists of the age of the person in
question and whether they have voted already.  We can represent this
as a Cartesian product of XML Schema~\cite{xmlschema} basic types,
as follows:
\begin{equation*}
  E = \mathcal{P} (\mtt{xsd:int}) \times \mathcal{P} (\mtt{xsd:boolean})
  \times \mathcal{P} (\mtt{xsd:string})
  \label{eqn:env}
\end{equation*}
The first component of the environment $E$ is the age of the
person, the second component is whether or not they have voted
already, and the third component is the action they are attempting
(perhaps voting, but perhaps something else).  We use power sets here
because in XACML all attributes describe sets of values, never
singletons.

The XACML policy for this
example is shown in Figure \ref{fig:example}.
We will explain the semantics of this policy
using a simple mathematical notation below. 
The goal for our policy is that if a person is doing something other than
voting, we do not really care what happens, and we require that there
be only one age and one voting record presented.  To do this we can
divide $E$ into four sets, $E_a$, $E_v$, $E_p$ and $E_d$ as follows
(note that the notation $\exists ! \, x \, P$ asserts that there is a
unique $x$ that satisfies a condition $P$):
\begin{eqnarray*}
  E_a & = & \{ \langle a, v, o \rangle \in E : \exists ! \, a_0 \in a 
  \wedge \exists ! \, v_0 \in v \} \\
  E_v & = & \{ \langle a, v, o \rangle \in E_a : \exists x \in o \; 
  x = \texttt{vote} \} \\
  E_p & = & \{ \langle \{a_0\}, \{v_0\}, o \rangle \in E_v : a_0 \geq 18 
  \wedge \neg v_0 \} \\
  E_d & = & E_v - E_p 
   =  \{ \langle \{a_0\}, \{v_0\}, o \rangle \in E_v : a_0 < 18 \vee v_0 \}
\end{eqnarray*}
Here, $E_a$ is the set of all environments whose inputs are not
erroneous, $E_v$ is the set of all environments where voting is
attempted, $E_p$ is the set of all environments where the person can
vote (their attempt to vote is \emph{permitted}), and $E_d$ is the set
of all environments where the person cannot vote (their attempt to
vote is \emph{denied}).  
In the following
section we will define a concise formal model for XACML policies.

\subsection{Formal Model}
\label{sec:formal-model}

Let $R = \{ \Permit, \Deny, \NotApplicable, \Indeterminate \}$ be the set
of valid results.  Now, we can define the set of valid policies $P$ as
follows (semantics will be defined later):
\begin{eqnarray*}
  \Permit & \in & P \\
  \Deny & \in & P \\
  \forall p \in P \;\, \forall S \subseteq E \; \Scope (p, S) & \in & P \\
  \forall p \in P \;\, \forall S \subseteq E \; \Err (p, S) & \in & P \\
  \forall p, q \in P \; p \oplus q & \in & P \\
  \forall p, q \in P \; p \ominus q & \in & P \\
  \forall p, q \in P \; p \otimes q & \in & P \\
  \forall p, q \in P \; p \oslash q & \in & P
\end{eqnarray*}
Informally, we regard \Permit{} and \Deny{} as symbols whose semantics
ignore the environment, always yielding \Permit{} or \Deny{},
respectively.  Along these same lines, \Scope{} and \Err{} attach
conditions to policies:
\begin{itemize}
\item $\Scope (p, S)$ modifies policy $p$ to yield $p$'s
answer if the current environment is in $S$, or \NotApplicable{}
otherwise.
\item $\Err (p, S)$ yields \Indeterminate{} if 
the current environment is in $S$ or $p$'s answer otherwise.  
\end{itemize}
The other four
symbols ($\oplus, \ominus, \otimes, \oslash$) are combinators, that
combine two policies in various ways:
\begin{itemize}
\item {\bf Permit-overrides:} $p \oplus q$ always yields
\Permit{} if either $p$ or $q$ yield \Permit{}.
\item  {\bf Deny-overrides:} $p \ominus q$ always
yields \Deny{} if either $p$ or $q$ yield \Deny{}.
\item  {\bf Only-one-applicable:} $p \otimes q$ requires that
one of $p$ or $q$ yield \NotApplicable{} and then yields the other
half's answer.
\item {\bf First-applicable:} $p \oslash q$ yields $p$'s answer unless that is
\NotApplicable{}, in which case it yields $q$'s answer.
\end{itemize}

More formally, we define a function $\eff : E \times P \rightarrow R$
that, given an environment and a policy produces a result.  We define
this function in Figure~\ref{fig:semantics} so that it corresponds to
our intuition about the desired properties, as described above.

\begin{figure}[t]
\centering
\begin{footnotesize}
\begin{eqnarray*}
  \eff & : & E \times P \rightarrow R \\
  \eff (e, \Permit) & = & \Permit \\
  \eff (e, \Deny) & = & \Deny \\
  \eff (e, \Scope (p, S)) & = & \left\{\begin{array}{ll}
  \eff (e, p) & \text{if $e \in S$} \\
  \NotApplicable & \text{otherwise}
  \end{array}\right. \\
  \eff (e, \Err (p, S)) & = & \left\{\begin{array}{ll}
  \Indeterminate & \text{if $e \in S$} \\
  \eff (e, p) & \text{otherwise}
  \end{array}\right. \\
  \eff (e, p \oplus q) & = & \left\{\begin{array}{ll}
  \Permit & \text{if $\eff(e, p) = \Permit$} \\
  & \text{$\quad \vee \eff(e, q) = \Permit$} \\
  \Indeterminate & \text{if } (\eff(e, p) = \Indeterminate \\
  & \quad \wedge \eff(e, q) \neq \Permit) \\
  &  \vee (\eff(e, q) = \Indeterminate \\
  & \quad \wedge \eff(e, p) \neq \Permit) \\
  \Deny & \text{if } (\eff(e, p) = \Deny \\
  & \quad \wedge \eff(e, q) \neq \Permit \\
  & \quad \wedge \eff(e, q) \neq \Indeterminate) \\
  & \vee (\eff(e, q) = \Deny \\
  & \quad \wedge \eff(e, p) \neq \Permit \\
  & \quad \wedge \eff(e, p) \neq \Indeterminate) \\
  \NotApplicable & \text{otherwise}
  \end{array}\right. \\
  \eff (e, p \ominus q) & = & \left\{\begin{array}{ll}
  \Deny & \text{if $\eff(e, p) = \Deny$} \\
  & \text{$\quad \vee (\eff(e, q) = \Deny$} \\
  \Indeterminate & \text{if } (\eff(e, p) = \Indeterminate \\
  & \quad \wedge \eff(e, q) \neq \Deny) \\
  & \vee (\eff(e, q) = \Indeterminate \\
  & \quad \wedge \eff(e, p) \neq \Deny) \\
  \Permit & \text{if } (\eff(e, p) = \Permit \\
  & \quad \wedge \eff(e, q) \neq \Deny \\
  & \quad \wedge \eff(e, q) \neq \Indeterminate) \\
  & \vee (\eff(e, q) = \Permit \\
  & \quad \wedge \eff(e, p) \neq \Deny \\
  & \quad \wedge \eff(e, p)\neq \Indeterminate) \\
  \NotApplicable & \text{otherwise}
  \end{array}\right. \\
  \eff (e, p \otimes q) & = & \left\{\begin{array}{ll}
  \eff (e, p) & \text{if $\eff(e, q) = \NotApplicable$} \\
  \eff (e, q) & \text{if $\eff(e, p) = \NotApplicable$} \\
  \Indeterminate & \text{otherwise}
  \end{array}\right. \\
  \eff (e, p \oslash q) & = & \left\{\begin{array}{ll}
  \eff (e, p) & \text{if $\eff(e, p) \neq \NotApplicable$} \\
  \eff (e, q) & \text{otherwise}
  \end{array}\right.
\end{eqnarray*}
\end{footnotesize}
\caption{Semantics of policies}
\label{fig:semantics}
\end{figure}

Using this notation, we can now model our example as follows:
\begin{eqnarray}
  S_0 & = & \{ \langle a, v, o \rangle \in E : \forall x \in a \; x < 18 \} 
  \label{eq:s0} \\
  S_1 & = & \{ \langle a, v, o \rangle \in E : \forall x \in v \; x \}
  \label{eq:s1} \\
  S_2 & = & \{ \langle a, v, o \rangle \in E : 
  \exists x \in o \; x = \mtt{vote} \} \label{eq:s2} \\
  S_3 & = & \{ \langle a, v, o \rangle \in E : \neg \exists ! \, a_0 \in a \} 
  \label{eq:s3} \\
  S_4 & = & \{ \langle a, v, o \rangle \in E : \neg \exists ! \, v_0 \in v \}
  \label{eq:s4} \\
  r_1 & = & \Err (\Scope (\Deny, S_0), S_3) \label{eq:r1} \\
  r_2 & = & \Err (\Scope (\Deny, S_1), S_4) \label{eq:r2} \\
  p & = & \Scope (r_1 \ominus r_2 \ominus \Permit, S_2) \label{eq:p}
\end{eqnarray}
Here, $S_0$ is the set of environments that fail the age requirement,
$S_1$ is the set of environments that fail the voting requirement,
$S_2$ is the set of environments where someone's trying to vote, etc.

\subsection{Policy Transformations}
\label{sec:transformations}

\begin{figure}[t]
\centering
\begin{footnotesize}
\begin{eqnarray*}
  f & : & P   \rightarrow  P \nonumber \\
  f (\Scope (\Scope (X, S), R))  & = & f (\Scope (X, R \cap S)) 
  \nonumber \\
  f (\Scope (\Err (X, S), R)) & = & f (\Err (\Scope (X, R \setminus S), S \cap R)) 
  \nonumber \\
  f (\Scope (X \oplus Y, S)) & = & \Scope (f (X), S) \oplus \Scope (f (Y), S) 
  \nonumber \\
  f (\Scope (X \ominus Y, S)) & = & \Scope (f (X), S) \ominus \Scope (f (Y), S) 
  \nonumber \\
  f (\Scope (X \otimes Y, S)) & = & \Scope (f (X), S) \otimes \Scope (f (Y), S)) 
  \nonumber \\
  f (\Scope (X \oslash Y, S)) & = & \Scope (f (X), S) \oslash \Scope (f (Y), S) 
  \nonumber \\
  f (\Scope (P, S)) & = & \Scope (f (P), S) 
     \ \text{if no other rules apply} \nonumber \\
  f (\Err (\Err (X, S), R)) & = & f (\Err (X, R \cup S)) \nonumber \\
  f (\Err (\Scope (X, S), R)) & = & f (\Err (\Scope (X, S \setminus R), R)) 
     \ \text{if $S \cup R \neq \emptyset$} \nonumber \\
  f (\Err (X \oplus Y, S))  & = &  \Err (f (X), S) \oplus \Err (f (Y), S) \nonumber \\
  f (\Err (X \ominus Y, S)) & = &  \Err (f (X), S) \ominus \Err (f (Y), S) \nonumber \\
  f (\Err (X \otimes Y, S)) & = &  \Err (f (X), S) \otimes \Err (f (Y), S) \nonumber \\
  f (\Err (X \oslash Y, S)) & = &  \Err (f (X), S) \oslash \Err (f (Y), S) \nonumber \\
  f (\Err (P, S)) & = & \Err (f (P), S) 
     \  \text{if no other rules apply} \nonumber \\
  f (\Permit) & = & \Permit \nonumber \\
  f (\Deny) & = & \Deny \nonumber
\end{eqnarray*}
\end{footnotesize}
\caption{$\eff$-preserving transformations for reduction to normal form}
\label{fig:1streduction}
\end{figure}

We would like to perform analysis on this model, and it would be
easier to do this analysis if we could bring the model into a normal
form.  To do this, first we define equivalence:
\begin{equation*}
  P_1 \equiv P_2 \text{ iff } \forall e \in E \; \eff (e, P_1) = \eff (e, P_2)
\end{equation*}
We call a function $f$ that takes a policy and returns another
policy an \emph{$\eff$-preserving transformation} if $\forall p \in P
\; f (p) \equiv p$.

For any given policy, we want to regard the subset of $E$ that will
give a \Permit{} result, the subset of $E$ that will give a \Deny{}
result, and the subset of $E$ that will give an \Indeterminate{} result
independently.  We define the shorthand $\langle S, R, T \rangle$,
where $S$, $R$ and $T$ are pairwise disjoint, as follows:
\begin{equation*}
  \langle S, R, T \rangle = \Err (\Scope (\Permit, S) \otimes \Scope (\Deny, R), T)
\end{equation*}
Hence, $\langle S, R, T \rangle$ is simply a policy
that yields \Permit{} for any environment in $S$,
\Deny{} for any environment in $R$,
\Indeterminate{} for any environment in $T$,
and \NotApplicable{} for any remaining environment.
We call this \emph{triple notation} and refer to individual nodes
$\langle S, R, T \rangle$ as \emph{triples}.

Now that we have a framework for transforming policies, we would like
to transform an entire policy with $\Scope$, $\Err$ and combinators alike
into a single triple.  We know that for any policy $P$ a triple $P_T$
that is equivalent to it exists: the triple is just 
\begin{eqnarray*}
P_T & = & \langle \{
e \in E : \eff (e, P) = \Permit \}, \\
&& \;\, \{ e \in E : \eff (e, P) = \Deny \}, \\
&& \;\, \{ e \in E : \eff (e, P) = \Indeterminate \}\rangle.
\end{eqnarray*}
However,
this is not a constructive definition.  To transform the policies to
the triple form, we define two functions 
$f : P   \rightarrow  P$ and $g : P \rightarrow \langle S, R, T \rangle$, both
$\eff$-preserving transformations, such that $g (f(p))$ is a triple
for all $p$.  The $f$ function transforms the policy into an
equivalent one that is composed of triples joined by combinators.  The
$g$ function combines triples joined by combinators into a single
triple.  The two together generate the triple representation.  We
define $f$ in Figure~\ref{fig:1streduction}, and $g$ in
Figure~\ref{fig:triplereduction}.

As an example, applying $f$ to the policy $p$ defined in
Equation~\eqref{eq:p} leads to the following:
\begin{eqnarray*}
  p & = & \Scope (\Err (\Scope (\Deny, S_0), S_3) \\
  & & \qquad \ominus \, \Err (\Scope (\Deny, S_1), S_4) \\
  & & \qquad \ominus \, \Permit, S_2) \nonumber \\
  f(p)  & = & \Err (\Scope (\Deny, (S_2 \cap S_0) \setminus S_3), S_3 \cap S_2) \\
  & & \qquad \ominus \,
  \Err (\Scope (\Deny, (S_2 \cap S_1) \setminus S_4), S_4 \cap S_2) \\
  & & \qquad \ominus \, \Scope (\Permit, S_2)
\end{eqnarray*}
Note that the function $f$ pushes all $\Scope$ forms down to the leaves
of the policy tree, and all $\Err$ forms down to just above the leaves.

The $f$ function transforms a policy to a collection of expressions of
the form $\Err (\Scope (A, B), T)$ (where $A \in \{ \Permit, \Deny
\}$, $B, T \subseteq E$, and $B \cap T = \emptyset$) combined using
$\oplus, \ominus, \otimes$ and $\oslash$.  Since $\forall e \in E \,
\eff (e, X \otimes \Scope (Y, \emptyset)) = \eff (e, X)$, we can
further rewrite these expressions in the form $\Err (\Scope (\Permit,
S)$ $\otimes$ $\Scope (\Deny, R), T)$ combined with $\oplus, \ominus,
\otimes$ and $\oslash$ where $S = B$ and $R = \emptyset$ if $A =
\Permit$ and $S = \emptyset$ and $R = B$ if $A = \Deny$.  Since $S, R$
and $T$ are all pairwise disjoint this is exactly the required form
for our triple notation.  Hence, after applying the function $f$ we
have a set of subpolicies in our triple notation combined with
$\oplus, \ominus, \otimes$ and $\oslash$.  We define the function $g$
in Figure~\ref{fig:triplereduction}.  The transformations for function
$g$ all preserve the disjointness property, and using the function $g$
we can transform the policy generated by function $f$ to a single
triple $\langle S, R, T \rangle$ for some $S, R, T \subseteq E$.

\begin{figure*}[t]
\centering
\begin{footnotesize}
\begin{eqnarray*}
  g & : & P \rightarrow \langle S, R, T \rangle \\
  g (\langle S_1, R_1, T_1 \rangle \oplus \langle S_2, R_2, T_2 \rangle) & = & 
  \langle 
  S_1 \cup S_2, (R_1 \setminus (S_2 \cup T_2)) \cup (R_2 \setminus (S_1 \cup T_1)), 
  (T_1 \cup T_2) \setminus (S_1 \cup S_2)
  \rangle \\
  g (\langle S_1, R_1, T_1 \rangle \ominus \langle S_2, R_2, T_2 \rangle)
  & = & \langle (S_1 \setminus (R_2 \cup T_2)) \cup 
  (S_2 \setminus (R_1 \cup T_1)), 
  R_1 \cup R_2,
  (T_1 \cup T_2) \setminus (R_1 \cup R_2)
  \rangle
  \\
  g (\langle S_1, R_1, T_1 \rangle \otimes \langle S_2, R_2, T_2 \rangle)
  & = & \langle 
  (S_1 \cup S_2) \setminus ((S_1 \cap S_2) \cup T_1 \cup T_2), 
  (R_1 \cup R_2) \setminus ((R_1 \cap R_2) \cup T_1 \cup T_2), 
  T_1 \cup T_2 \cup (S_1 \cap S_2) \cup (R_1 \cap R_2)
  \rangle 
  \\
  g (\langle S_1, R_1, T_1 \rangle \oslash \langle S_2, R_2, T_2 \rangle)
  & = & \langle 
  S_1 \cup (S_2 \setminus (R_1 \cup T_1)),
  R_1 \cup (R_2 \setminus (S_1 \cup T_1)), 
  T_1 \cup (T_2 \setminus (S_1 \cup R_1))
  \rangle
  \\
  g (\langle S_1, R_1, T_1 \rangle) & = & \langle  S_1, R_1, T_1 \rangle 
  \\
  g (P_1 \oplus P_2) & = & g (g (P_1) \oplus g (P_2)) \quad 
  \text{if no other rules apply}  \\
  g (P_1 \ominus P_2) & = & g (g (P_1) \ominus g (P_2)) \quad 
  \text{if no other rules apply}  \\
  g (P_1 \otimes P_2) & = & g (g (P_1) \otimes g (P_2)) \quad 
  \text{if no other rules apply}  \\
  g (P_1 \oslash P_2) & = & g (g (P_1) \oslash g (P_2)) \quad
  \text{if no other rules apply} 
\end{eqnarray*}
\end{footnotesize}
\caption{$\eff$-preserving transformations for $\langle S, R, T \rangle$ 
  reduction}
\label{fig:triplereduction}
\end{figure*}

When we apply the function $g$ to our example we get the following:
\begin{eqnarray*}
  f(p) & = & \Err (\Scope (\Deny, (S_2 \cap S_0) \setminus S_3), S_3 \cap S_2) \\
  & & \ominus \, \Err (\Scope (\Deny, (S_2 \cap S_1) \setminus S_4), S_4 \cap S_2) \\
  & & \ominus \, \Scope (\Permit, S_2)
  \\
  & = & \langle \emptyset, S_2 \cap S_0 \setminus S_3, S_3 \cap S_2 \rangle \\
  & & \ominus \langle \emptyset, S_2 \cap S_1 \setminus S_4, S_4 \cap S_2 \rangle \\
  & & \ominus \langle S_2, \emptyset, \emptyset \rangle
  \\
  g(f(p))  & = & \langle S_2 \setminus (S_0 \cup S_1 \cup S_3 \cup S_4), \\
  & & ((S_0 \cup S_1) \setminus (S_3 \cup S_4)) \cap S_2, \\
  & &  (S_3 \cup S_4) \cap S_2 \rangle
\end{eqnarray*}

Now that we have our policy into a form that is convenient for
analysis, we can begin to prove things about it.

% arch-tag: 834f4d24-c6fd-4bef-9e73-89b1efa5ca0b

%%% Local Variables: 
%%% mode: latex
%%% TeX-master: "paper"
%%% End: 

\section{Properties of Policies}
\label{sec:properties-policies}

In this section we will 
show that properties of policies can be 
expressed based on several partial ordering relations.
For example, we might want to prove
that a (possibly very complex) policy at least protects as much as
some simpler policy, and similarly we might want to guarantee that a
(possibly very complex) policy does not say anything outside of its
scope. Such properties can be expressed using the ordering relations
defined below.

Let $P_1 = \langle S_1, R_1, T_1 \rangle$ and let $P_2 =
\langle S_2, R_2, T_2 \rangle$ be two policies.  
We define the following
partial orders:
\begin{eqnarray*}
  P_1 \sqsubseteq_P P_2 & \mbox{iff} & S_1 \subseteq S_2 \\
  P_1 \sqsubseteq_D P_2 & \mbox{iff} & R_1 \subseteq R_2 \\
  P_1 \sqsubseteq_E P_2 & \mbox{iff} & T_1 \subseteq T_2 \\
  P_1 \sqsubseteq_{P,D,E} P_2 & \mbox{iff} & P_1 \sqsubseteq_P P_2
  \wedge P_1 \sqsubseteq_D P_2 \wedge P_1 \sqsubseteq_E P_2 
\end{eqnarray*}
Note that, we can define a partial order for
for any combination of of $P$, $D$ and $E$.
We define $P_1 \sqsubseteq P_2 \equiv P_1 \sqsubseteq_{P,D,E} P_2$.
We can regard $P_1 \sqsubseteq P_2$ as stating that for any $e \in E$
where $\eff (P_1, e) \neq \NotApplicable$, $\eff (P_2, e) = \eff (P_1,
e)$.

To demonstrate the use of these ordering relations, 
let us create a new policy; people are permitted
to check the current results of the election, for exit polls.  
We encode this with the
following policy 
\begin{eqnarray*}
  S_5 & = & \{ \langle a, v, o \rangle \in E : 
  \exists x \in o \; x = \mtt{getresult} \} \\
  r_3 & = & \Scope (\Err (\Permit, S_4), S_5)
\end{eqnarray*}
where $S_4$ is defined in Equation~\eqref{eq:s4}.
Now, we can create a composite policy as follows $p_c = p \oplus r_3$,
where $p$ is defined in Equation~\eqref{eq:p}.  This
policy has a bug---specifically, it permits people under 18 to vote in
certain circumstances---and we will demonstrate the usefulness of our
technique by showing this.  First, we perform our translations on this
new policy as above, getting:
\begin{eqnarray*}
  g(f(r_3)) & = & \langle S_5 \setminus S_4, \emptyset, S_4 \cap S_5 \rangle \\
  g(f(p_c)) & = & \langle ((S_2 \setminus (S_0 \cup S_1 \cup S_3 \cup S_4)) 
  \cup (S_5 \setminus S_4)), \\
  & & \;\, (((S_0 \setminus S_3) \cup (S_1 \setminus S_4)) \cap S_2) 
  \setminus (S_4 \cap S_5), \\
  & & \;\, ((S_4 \cap S_5) \cup ((S_3 \cup S_4) \setminus \\
  & & \;\, \qquad ((S_0 \setminus S_3) \cup 
  (S_1 \setminus S_4))) \cap S_2) \setminus \\
  & & \;\, \quad ((S_2 \setminus (S_0 \cup S_1 \cup S_3 \cup S_4)) 
  \cup (S_5 \setminus S_4)) \rangle \\
\end{eqnarray*}
where $S_0,S_1, S_3$ and $S_4$ are from
Equations~\eqref{eq:s1} to ~\eqref{eq:s4}.
Using set algebra we can simplify the expression for policy $p_c$ to
\begin{align*}
  g(f(p_c)) = \big\langle & ((S_2 \setminus (S_0 \cup S_1 \cup S_3)) 
  \cup S_5) \setminus S_4, \\
  & (((S_0 \setminus S_3) \cup (S_1 \setminus S_4)) \cap S_2) 
  \setminus (S_4 \cap S_5), \\
  & \big((S_4 \cap S_5) \setminus (S_2 \setminus (S_0 \cup S_1 \cup S_3))\big)
  \cup \\
  & \quad \big(((S_3 \cup S_4) \setminus ((S_0 \setminus S_3) \cup
  (S_1 \setminus S_4))) \cap S_2 \big) \big\rangle
\end{align*}
Now, we insist that this combined policy deny anyone trying to vote
who is under 18.  This is itself a policy, which we call $p_v$:
\begin{align*}
  p_v = \langle \emptyset, (S_0 \cap S_2) \setminus (S_3 \cup S_4),
  (S_3 \cup S_4) \cap S_2 \rangle
\end{align*}
The interesting thing here is whether or not $p_v \sqsubseteq_D p_c$,
i.e., does the policy $p_c$ deny every input that is denied by $p_v$.
That would mean that everyone trying to vote who is under 18 is
denied, and that our policy combination has not done any harm.
However, the environmental tuple \[e = \langle \{17\}, \{\mtt{true}\},
\{\mtt{vote}, \mtt{getresult} \} \rangle\] demonstrates that that is
not the case.  Input $e$ passes the second part of the \Permit{}
requirement and so is permitted by $p_c$ (which means that it is {\em
  not} denied by $p_c$) but denied by $p_v$, i.e., $e$ demonstrates
that $p_v \not \sqsubseteq_D p_c$.  The error is that, we do not
enforce that only one action be given in the third component of the
input, and because of this we have the surprising result that someone
who is under eighteen and has already voted, but asks for the voting
results at the same time as trying to vote will be permitted, and so
can cast any number of ballots.  To fix this, we could insist upon a
new condition, that $\exists ! \, x \in o$; or we could use $\otimes$
instead of $\oplus$, which would ensure that only one of the
sub-policies could be definitive on any given point (and so turn $\eff
(e, p_v)$ into an \Indeterminate{} result instead of a \Permit{}); or we
could decide that only people who have voted already can check the
results.

% arch-tag: e2b0e59e-49da-4b4c-9860-1da5c30b891f

%%% Local Variables: 
%%% mode: latex
%%% TeX-master: "paper"
%%% End: 

\section{Automatically Proving Properties of Policies}
\label{sec:autom-prov-prop}

Given the formal model defined in Section~\ref{sec:formal-model} and
properties defined in Section~\ref{sec:properties-policies} we would
like to check properties of access policies automatically.  To do this
we first formalize the syntax of formulas we use to specify subsets of
$E$.  Then we discuss how policies constructed using these formulas
and policy combinators can be translated to the Alloy language.  One
can check properties of access policies by translating them to SAT
problems.  Instead of targeting a SAT problem directly, our translator
targets the Alloy language, which is in turn translated into a SAT
problem by the Alloy Analyzer.  This approach simplifies our
translator and permits us to remain closer to the syntax of the
original problem statement.

\subsection{Policy Classes}
\label{sec:policy-classes}

In Section~\ref{sec:formal-model}, we defined our formal model using
subsets of $E$ for the definition of \Scope and \Err.  Since $E$ is
the Cartesian product of the power sets of what are in most cases
infinite sets, analyzing these properties would seem to require
enumerating these subsets, an impossible job.  Fortunately, we are
translating from XACML to our foreign model, and totally arbitrary
subsets are not possible in XACML.  We can structure the subsets we
will get from XACML policies in a more useful fashion: they end up
looking like the sets $S_0, S_1, \dots, S_4$ in
Equation~\eqref{eq:s0}.

To formalize this, we fix the following notational conveniences: for
elements $e \in E$, we name the components of $e$ $e[0], e[1], \dots,
e[n]$.  We use $s, s_0, s_1, \dots, s_n$ to denote set variables, $c,
c_0, c_1, \dots, c_n$ to denote scalar variables, and $C, C_0, C_1,
\dots, C_n$ to denote constants.  Finally, $BP$ is a set of basic
predicates which we will define below.  We fix our structure in the
following fashion: all subsets of $E$ are specified in the form $\{ e
\in E : P \}$, where there are no free variables save $e$ in $P$ and
$P$ is defined as follows:
\begin{eqnarray*}
  P & \rightarrow & BP \quad | \quad \forall c \in s \; P \quad |
  \quad \exists c \in s \; P \\
  & | & \exists ! \, c \in s \; P \quad | \quad \exists ! \, c \in e[i] \; P \\
  & | & P \wedge P \quad | \quad P \vee P \quad | \quad \neg P
\end{eqnarray*}

Below, we will define four different basic predicate sets with
increasing expressive power, such that $BP_1 \subseteq BP_2 \subseteq
BP_3 \subseteq BP_4$.  The version of $BP$ that used is important for
translation. For example, if a policy can be written using only
predicates from $BP_1$ then we can make certain guarantees about its
translation that may not hold if we must simulate predicates from a
larger $BP$.

Our first class is that of policies using only enumerated types (which
obviously have finite domains) and the simple operations $\neg$, $=$,
$\in$, $\subseteq$.  We define the first class of basic predicates,
$BP_1$, as follows:
\begin{eqnarray*}
  SCAL & \rightarrow & c \quad | \quad C \\
  BSET & \rightarrow & s \quad | \quad e[i] \\
  SET & \rightarrow & SET \cup SET \quad | \quad SET \cap SET \\
  & | & SET \setminus SET \quad | \quad \{ SCAL \} \quad | \quad BSET \\
  BP_1 & \rightarrow & SET \subseteq SET \quad | \quad SCAL \in SET \\
  & | & SCAL = SCAL \quad | \quad \texttt{true} \quad | \quad \texttt{false}
\end{eqnarray*}
We can express all set definitions on unordered and enumerated types
that are permitted in XACML using the expressions above.

Given a set in the form $S = \{ e \in E : P \}$ where $P$ is defined
based on the above syntax, one can generate a boolean logic formula
$B$ which encodes the set $S$.  The encoding will map each $e \in E$
to a valuation of the boolean variables in $B$ and $B$ will evaluate
to true if and only if $e \in S$.  Based on such an encoding we can
convert questions about different policies (such as if one subsumes
the other one) to SAT problems and then use a SAT solver to check
them.  For example, we can generate a boolean formula which is
satisfiable if and only if an access policy is subsumed (i.e.,
$\sqsubseteq$) by another one.  If the SAT solver returns a satisfying
assignment to the formula, then we can conclude that the property is
false, and generate a counterexample based on the satisfying
assignment.  If the SAT solver declares that the formula is not
satisfiable then we can conclude that the property holds.  Details of
such a translation for the Alloy language is given
in~\cite{jackson00automating}, and as we will show below the policies
specified with the syntax described above can be translated to Alloy
language.

The second class of basic predicates extends the first one to handle
types which have a total order relation $<$, as well.  We define
$BP_2$ as follows:
\begin{eqnarray*}
 BP_2 & \rightarrow & BP_1 \quad | \quad SCAL < SCAL 
\end{eqnarray*}
Sets described using this class of predicates can also be translated
to a boolean logic formula. We can encode a type with a domain of $n$
ordered elements using $n^2$ boolean variables, one for each pair of
values in the domain.

The third class of policies extends the second class to include
infinite domains. Although the syntax for $BP_2$ and $BP_3$ are the
same for $BP_3$ we permit $E$ to be composed of power sets of infinite
domains.  Note that we cannot translate sets described using such
predicates directly to boolean logic formulas.  So instead we limit
the scope of our investigations: we artificially limit the size of the
set to a given fixed size and then perform analysis upon it as though
it were a finite enumerated set of that size. The problem is that if
no counterexample is found, then that does not necessarily mean that
no counterexample exists---perhaps if we had increased the scope just
a little more we would have found one. The small scope hypothesis
(discussed in~\cite{jackson:elements}, and tested and confirmed for
some data structure algorithms in~\cite{marinov:exhaustive}) suggests
that small scopes could be sufficient in practice. Note that if a
counterexample is found, that counterexample is definite and can be
translated into an error in the original policy.

The fourth and final class of policies extends the third class to
handle arbitrary Boolean functions, with any scalars or sets as
arguments. We also handle one special function on sets, that being the
magnitude operation.
\begin{eqnarray*}
  BP_4 & \rightarrow & BP_3 \quad | \quad f (SET, \dots, SCAL, \dots)
\quad | \quad | SET | 
\end{eqnarray*}
To translate sets described using such predicates to boolean logic
formulas we use uninterpreted functions, i.e., we create a Boolean
variable for encoding the value of a boolean function and we create a
(bounded) integer variable for encoding the size of set. We generate
constraints which guarantee that the value of the function is the same
if its arguments are the same. Other than this restriction the
variables encoding the functions can get arbitrary values.  Note that
this brings an extra level of imprecision to our analysis.  We were
not able to trust the positive results because of the scope
restriction, but now it is also possible that counterexamples may be
spurious, and will need to be validated against the original policy.
However, we think that such automated analysis can still be useful in
uncovering errors in access policies.

The above suffices for modeling every function in core XACML, with the
exception of the higher order functions.  Those functions invariably
operate on sets, complicate the analysis, and---if these functions are
restricted to using the predicates that we can model directly---can
ultimately be expressed using the predicates we can already handle.

In the above we defined the syntax for four policy classes and argued
that properties of policies described with the above syntax can be
translated to SAT problems.  Our analysis tool instead targets the
modeling language
Alloy~\cite{jackson:alcoa,jackson:micromodels,jackson00automating}.
Alloy permits a more literal translation of our model, simplifying the
translation tremendously. After introducing Alloy briefly, we will
show how we translate our four classes of policies to Alloy.

\subsection{Alloy}

Alloy is a declarative modeling language out of MIT equipped with an
analyzer that can verify assertions about models written in the
language.  Alloy analyzer achieves this by converting assertions to
Boolean logic formulas which are fed to a SAT solver.  Alloy is based
in first order relational logic, and is intended to model complex
structures. It does so through extensive set manipulation, and this
manipulation permits an easy translation from our mathematical model.
Alloy has been used to automatically extract object
models~\cite{jackson01lightweight,waingold01automated}, to analyze the
behavior of filesystem synchronization
utilities~\cite{nolte02filesystem}, to model virtual
functions~\cite{marinov02valloy} and to automatically check structural
properties of data on the heap~\cite{vaziri03heap}.

Alloy models consist of sets of concrete objects, called
\emph{signatures}, facts about these sets, and relations on these
sets.  Distinguished subsets of signatures are possible; these new
signatures are said to \emph{extend} the superset.  Unlike some other
modeling languages, Alloy does not require that these relations be
completely specified.  After defining signatures and facts about them,
one can ask Alloy to verify that certain properties hold in all
possible models that conform to the facts given, or that there exists
a model capable of satisfying all the facts given.  Alloy cannot, in
general, prove assertions about all possible models; it can, however,
prove assertions for all models within a fixed scope, which is what we
have to settle for analyzing access policies in general as well.

One oddity about Alloy is that it unifies singleton sets and scalars;
this is done for technical reasons, but it has some implications for
our translation that will be discussed in the next section as they
arise.

\subsection{Translation to Alloy}
\label{sec:translation-alloy}

The general structure we will be using here is as follows: to prove
that $P_1 \sqsubseteq P_2$, we need to prove that each individual
component of $P_1$ is a subset of each individual component of $P_2$.
This part of the generated Alloy code is as follows:
\begin{verbatim}
static sig P1 extends Triple {} {
    ...
}
static sig P2 extends Triple {} {
    ...
}
assert Subset {
    P1.permit in P2.permit
    P1.deny in P2.deny
    P1.error in P2.error
}
\end{verbatim}
That is, we define two models \texttt{P1} and \texttt{P2}, and then
check that the components of the one are contained in the other.
Since Alloy unifies sets and singletons, \texttt{in} can do double
duty as set membership and subset testing.  Similarly we do not need
to specifically handle the conversion of scalar variables to singleton
sets.

We can translate $P_1$ and $P_2$ in our mathematical model in the
following manner. First, we distinguish predicates outside of
existential and universal quantifiers from predicates inside: that is,
we split $P$ into $P'$ and $P''$ as follows:
\begin{eqnarray*}
  P' & \rightarrow & BP \quad | \quad \forall c \in s \; P'' \quad |
  \quad \exists c \in s \; P'' \\
  & | & \exists ! \, c \in s \; P'' \quad | 
  \quad \exists ! \, c \in e[i] \; P'' \\
  & | & P' \wedge P' \quad | \quad P' \vee P' \quad | \quad \neg P'' \\
  P'' & \rightarrow & BP \quad | \quad \forall c \in s \; P'' \quad |
  \quad \exists c \in s \; P'' \\
  & | & \exists ! \, c \in s \; P'' \quad | \quad 
  \exists ! \, c \in e[i] \; P'' \\
  & | & P'' \wedge P'' \quad | \quad P'' \vee P'' \quad | \quad \neg P''
\end{eqnarray*}
We translate tuples according to the rules in
Figure~\ref{fig:alloy-translation}, and some of these tuples will
create auxiliary sets much like the sets $S_0, S_1, S_2, S_3$
(Equations ~\eqref{eq:s1} to ~\eqref{eq:s4}) we used in
Section~\ref{sec:formal-model}.  The function $\extract$ defines a new
subset of the environment \texttt{E} based on its argument (which is a
formula) and then returns the name of this subset.  So $\extract(e)$
would return \texttt{S$_i$} and generate the following definition:
\verb|sig S|$_i$ \verb|extends E {} {| $e$ \verb|}|.

\begin{figure}[t]
\[
\begin{array}{l}
 \translation(\langle r, s, t \rangle) \Rightarrow
  \begin{aligned}
    \mtt{permit = } \translation_{P'} (r) \\
    \mtt{deny = } \translation_{P'} (s) \\
    \mtt{error = } \translation_{P'} (t)
  \end{aligned} \nonumber \\
  \\
  \mbox{For predicates outside quantifier formulas} \\
  \translation_{P'} (P'_1 \cap P'_2) \Rightarrow  
  \translation_{P'} (P'_1) \mtt{ \& } \translation_{P'} (P'_2) \nonumber \\
  \translation_{P'} (P'_1 \cup P'_2) \Rightarrow  
  \translation_{P'} (P'_1) \mtt{ + } \translation_{P'} (P'_2) \nonumber \\
  \translation_{P'} (\neg P'_1) \Rightarrow  
  \mtt{E - } \translation_{P'} (P'_1) \\ % \label{eq:6} \\
  \translation_{P'} (\forall c \in s \; P'') \Rightarrow 
  \extract(\mtt{all } c \mtt{:} \translation_{P''} (s)  \\
   \qquad \qquad \qquad \qquad \qquad
              \mtt{ | } \translation_{P''} (P'')) \\ %\label{eq:2} \\
  \translation_{P'} (\exists c \in s \; P'') \Rightarrow 
  \extract(\mtt{some } c \mtt{:} \translation_{P''} (s) \\
  \qquad \qquad \qquad \qquad \qquad \mtt{ | }
  \translation_{P''} (P'')) \nonumber \\
  \translation_{P'} (\exists ! \, c \in s \; P'') \Rightarrow 
  \extract(\mtt{one } c \mtt{:} \translation_{P''} (s) \\
  \qquad \qquad \qquad \qquad \qquad \mtt{ | }
  \translation_{P''} (P'')) \nonumber \\
  \translation_{P'} (BP) \Rightarrow  \translation_{BP} (BP) \nonumber \\
  \\
  \mbox{For predicates inside quantifier formulas} \\
  \translation_{P''} (P''_1 \cap P''_2) \Rightarrow  
  \translation_{P''} (P''_1) \mtt{ \&\& } \translation_{P''} (P''_2) 
  \nonumber \\
  \translation_{P''} (P''_1 \cup P''_2) \Rightarrow  
  \translation_{P''} (P''_1) \mtt{ || } \translation_{P''} (P''_2) \nonumber \\
  \translation_{P''} (\neg P''_1) \Rightarrow  
  \mtt{!} \translation_{P''} (P''_1) \nonumber \\
  \translation_{P''} (\forall c \in s \; P'') \Rightarrow 
  \extract(\mtt{all } c \mtt{:} \translation_{P''} (s) \\
   \qquad \qquad \qquad \qquad \qquad \mtt{ | }
  \translation_{P''} (P'')) \\ % \label{eq:1} \\
  \translation_{P''} (\exists c \in s \; P'') \Rightarrow 
  \extract(\mtt{some } c \mtt{:} \translation_{P''} (s) \\
   \qquad \qquad \qquad \qquad \qquad \mtt{ | }
  \translation_{P''} (P'')) \nonumber \\
  \translation_{P''} (\exists ! \, c \in s \; P'') \Rightarrow 
  \extract(\mtt{one } c \mtt{:} \translation_{P''} (s) \\
  \qquad \qquad \qquad \qquad \qquad \mtt{ | }
  \translation_{P''} (P'')) \nonumber \\
  \translation_{P''} (BP) \Rightarrow  \translation_{BP} (BP) \nonumber
\end{array}
\]
\caption{Basic translation rules}
\label{fig:alloy-translation}
\end{figure}

Now, we just need to know how to translate the $BP$ hierarchy to
Alloy.  Translating the various sorts of basic predicates in $BP_1$ is
straightforward but has minor complications: Alloy equates scalar
quantities and sets with only one element. So, $\subseteq$ is the same
operation as $\in$. To create constant elements, we create a field in
a structure composed entirely of static constants which looks like:
\begin{verbatim}
static sig CONSTANTS {
    x1 : scalar Integer,
    x2 : scalar Integer,
    ...
}
\end{verbatim}
We use $\constant (C)$ to describe the operation that inserts the
constant $C$ into this table if it is not already present, and returns
a name to refer to it (for example, \texttt{CONSTANTS.x1} for the
first field).  Since the Boolean constants \texttt{True} and
\texttt{False} are already defined in Alloy we do not need to go
through this operation for Boolean constants, and can translate those
directly.

The $e[i]$'s are represented as fields of the \texttt{E} structure
which is declared as:
\begin{verbatim}
sig E {
    age : set Integer,
    voted : set Bool,
    actions : set String
}
\end{verbatim}
for the environment $E$ defined in Equation~(\ref{eqn:env}) for our
running example.  We use $\env (e[i])$ to give the translation for the
environmental set $e[i]$. For the running example, $\env (e[0]) =
\mtt{age}$ and $\env (e[2]) = \mtt{actions}$.

The total translation for $BP_1$ can be done as follows:
\begin{align}
  \translation_{BP} (s) \Rightarrow & s \nonumber \\
  \translation_{BP} (c) \Rightarrow & c \nonumber \\
  \translation_{BP} (e[i]) \Rightarrow & \env (e[i]) \nonumber \\ 
              % \label{eq:5} \\
  \translation_{BP} (C) \Rightarrow & \constant (C) \nonumber \\
  \translation_{BP} (s_i \cup s_j) \Rightarrow & \translation(s_i) \mtt{ + }
  \translation_{BP} (s_j) \nonumber \\
  \translation_{BP} (s_i \cap s_j) \Rightarrow & \translation(s_i) \mtt{ \& }
  \translation_{BP} (s_j) \nonumber \\
  \translation_{BP} (s_i \setminus s_j) \Rightarrow & \translation(s_i)
  \mtt{ - } \translation_{BP} (s_j) \nonumber \\
  \translation_{BP} (\{ c \}) \Rightarrow & \translation(c) \nonumber \\
  \translation_{BP} (c_i = c_j) \Rightarrow & \translation(c_i) \mtt{ = }
  \translation_{BP} (c_j) \nonumber \\
  \translation_{BP} (c \in s) \Rightarrow & \translation(c) \mtt{ in }
  \translation_{BP} (s) \nonumber \\
  \translation_{BP} (s_i \subseteq s_j) \Rightarrow & \translation(s_i)
  \mtt{ in } \translation_{BP} (s_j) \nonumber
\end{align}
Using all of this, we can show as an example, a translation of $S_1$
in Equation~(\ref{eq:s1}) into Alloy which results in:
\begin{verbatim}
sig S1 extends E {} { all x : voted | x = False }
\end{verbatim}
For policies of the second class, we add the predicate $<$.  To
accommodate this, we define an Alloy function \texttt{LessThan} and
enforce its transitivity as follows:
\begin{verbatim}
fact {
    all a,b,c:Type {
        LessThan (a, b) = True && 
        LessThan (b, c) = True => 
            LessThan (a, c) = True
    }
}
\end{verbatim}
Now we can simply translate
\begin{multline*}
  \translation(a < b) \Rightarrow \\
  \mtt{LessThan} (\translation(a), \translation(b)) \mtt{ = True}
\end{multline*}
As an example, we translate $S_0$ in Equation~(\ref{eq:s0}) which
looks like:
\begin{verbatim}
static sig CONSTANTS {
    x1 : scalar Integer
}
sig S1 extends E {} { all x : age | 
    LessThan (a, CONSTANTS.x1) = True }
\end{verbatim}
Policies of the third class require no special translation; they
merely require that Alloy be informed of the scope requirements when
it attempts to analyze the policy.

Policies of the fourth class are accommodated by defining a new Alloy
relation about which we specify nothing.  For example, suppose we
wanted to analyze a policy involving XACML's embedded XPath matching.
Since we will encode this as an uninterpreted function all we need to
know about XPath matching is that it returns a Boolean value. We
define a set $S_6$ as follows:
\begin{equation*}
  S_6 = \{ \langle a, v, o \rangle \in E : \mtt{xpathnodematch}
  (\mtt{/actions}, o) \}
\end{equation*}
where {\tt /actions} is an XPath expression.  We translate this by
first introducing a new function as follows:
\begin{verbatim}
static sig Functions {
    // xpathnodematch(/actions, o)
    expr1 : E -> Bool
}
\end{verbatim}
and we use it in a new subset of \texttt{E} as follows:
\begin{verbatim}
sig S6 extends E {} {
    this.(Functions.expr1) = True
}
\end{verbatim}

We make no other claims about \texttt{Functions.expr1}, and as a
result \texttt{S6} represents an arbitrary subset of \texttt{E}.

% arch-tag: 931ea1c1-9ebf-4e94-89e7-06ac1ad3856f

%%% Local Variables: 
%%% mode: latex
%%% TeX-master: "paper"
%%% End: 

\section{Experiments}
\label{sec:experiments}

Our tool generates Alloy code which is then run through the Alloy
Analyzer to do the analysis.  It is targeted for proving things about
$\sqsubseteq$ relations, but we can use it for simpler questions as
well.

One such question might be `give a tuple $e$ such that $\eff (e, p) =
\Permit$' (where $p$ is defined as in our running example).  We
generate the Alloy code for the XACML policy, as normal, and then
append the following:
\begin{verbatim}
fun CheckTuple {
    some T0.permit
}
run CheckTuple for 2 but 2 Bool, 1 Triple, 8 Type
\end{verbatim}
The numbers after \texttt{CheckTuple} establish how deeply we will
look; we're looking for such a tuple in a universe where there are two
Boolean values, only one Triple (\texttt{T0}, the one generated
through $\translation(p)$), 8 domain types (strings, integers, and the
like)), and two elements of everything else.  If we run this, the
analyzer tells us that the tuple $\langle \{ \mtt{Type\_6} \},$ $\{
\mtt{Bool\_1} \},$ $\{ \mtt{Type\_7} \} \rangle$ generates a
\Permit{}, and further examination of the output shows that
\texttt{Bool\_1} is \texttt{True}, \texttt{Type\_6} is 18, and
\texttt{Type\_7} is the string \texttt{vote}.  Through similar means
we can discover that the tuple $\langle \{ 18 \}, \{ \}, \{ \mtt{vote}
\} \rangle$ will generate an error.

For something more interesting, we try to show that $p_v \sqsubseteq_D
p_c$, as we proved manually.  Our coda now becomes
\begin{verbatim}
assert Subset {
    T0.deny in T1.deny
}
check Subset for 2 but 2 Bool, 2 Triple, 10 Type
\end{verbatim}
We get a counterexample almost immediately, giving us the tuple we
constructed in Section~\ref{sec:properties-policies}.  If we modify
the policy so as to restrict result checking to only those who voted
successfully, then the subset relation holds, and no counterexample is
produced.

\subsection{Timing Data}

These are but small examples, so they do not reflect the time required
to solve large problems that one might reasonably ask.  Since the
underlying problem (SAT) is NP complete, it is reasonable to ask
whether these techniques are useful at all as the problem size
increases.

There are two sides to this.  One side is that larger and more complex
policies will inescapably take longer than small policies.  The other
is that---in the context of problems in $BP_3$ and $BP_4$, where we
must restrain the domains to a certain finite size---the amount of
computation involved as the size of the domain increases may trigger
the exponential worst case behavior.

\begin{figure}
  \centering
  \includegraphics{data/chart}
  \caption{Run time vs. domain size plotted for $P_1$ and $P_2$}
  \label{fig:graph}
\end{figure}

\begin{table}
\centering
\begin{tabular}{ccc}
Domain size & \multicolumn{2}{c}{Comparison} \\
& $P_3 \sqsubseteq P_4$ & $P_4 \not\sqsubseteq P_3$ \\
\input{data/invloutput}
\end{tabular}
\caption{Median run time of Medico example ($P_4$) and subset ($P_3$) in seconds}
\label{tbl:bigdata}
\end{table}

To demonstrate that the analysis is still feasible, we have collected
two sets of data.  The first, which we have charted in
Figure~\ref{fig:graph}, shows the time required to verify or refute a
relationship between two example policies.  It shows how the technique
scales as the size of the domain sets increases.  The second set of
data proves and refutes a relationship between the Medico policy from
Section~4.2 of the XACML specification~\cite{xacml} and a subset of
itself.  This shows that the technique is feasible for larger
policies.  All these benchmarks were performed on a 1 GHz PowerPC, and
all times are the median of five runs, to smooth out irregularities.
In each case, trials were run until the formula proved too large for
Alloy Analyzer to handle; past the given sizes the analyzer would fail
cryptically.  ``Domain size'' means the number of elements we are
using for our analysis, in each domain; so a domain size of 8 for our
example environment $E$ means we simulate every $e \in E$ where
$|e[i]| \leq 8$ for each component in $E$.

The data indicates that the time required for analysis is exponential
in the size of the scope, which is to be expected for a SAT based
algorithm.  However, all times are under two minutes, and our
technique can clearly prove important properties of these problems in
a useful amount of time.

% arch-tag: 5a3a4d97-641f-4c94-804d-9ec83256e4e6

%%% Local Variables: 
%%% mode: latex
%%% TeX-master: "paper"
%%% End: 

\section{Conclusion}

We have presented a formal model for access policies, and shown how to
analyze interesting properties about such models in an automated way.
We have implemented a tool to translate XACML policies into this model
and to yield code suitable for analysis.  The experimental results
indicate that automated analysis of nontrivial access policies is
feasible.

It would be interesting to investigate using predicate abstraction to
generate more precise models for the functions that we cannot directly
simulate.  Translating directly to a SAT solver may be more efficient
than going through Alloy, but we would have to try it to be sure.  We
would also like to experiment on more and larger policies.

% arch-tag: d0400f29-ab24-4a88-a786-3e22d798dc41

%%% Local Variables: 
%%% mode: latex
%%% TeX-master: "paper"
%%% End: 


\bibliographystyle{abbrv}
%\bibliography{citations/citations.bib}
\begin{thebibliography}{10}

\bibitem{abadpeiro99plas}
J.~L. Abad-Peiro, H.~Debar, T.~Schweinberger, and P.~Trommler.
\newblock {PLAS} --- {P}olicy language for authorizations.
\newblock Technical Report RZ 3126, IBM Research Division, 1999.

\bibitem{abadi93calculus}
M.~Abadi, M.~Burrows, B.~Lampson, and G.~Plotkin.
\newblock A calculus for access control in distributed systems.
\newblock {\em ACM Transactions on Programming Languages and Systems},
  15(4):706--734, September 1993.

\bibitem{bertino98access}
E.~Bertino, C.~Bettini, E.~Ferrari, and P.~Samarati.
\newblock An access control model supporting periodicity constraints and
  temporal reasoning.
\newblock {\em ACM Transactions on Database Systems}, 23(3):231--285, 1998.

\bibitem{bonatti01dataarchives}
P.~Bonatti, E.~Damiani, S.~{De Capitani di Vimercati}, and P.~Samarati.
\newblock An access control model for data archives.
\newblock In {\em Proceedings of the 16th international conference on
  information security: trusted information}, pages 261--276, Paris, France,
  2001. Kluwer International Federation For Information Processing Series.

\bibitem{bonatti02algebra}
P.~Bonatti, S.~{De Capitani di Vimercati}, and P.~Samarati.
\newblock An algebra for composing access control policies.
\newblock {\em ACM Transactions on Information Systems Security}, 5(1):1--35,
  2002.

\bibitem{damiani02SVG}
E.~Damiani, S.~{De Capitani di Vimercati}, E.~Fern\'andez-Medina, and
  P.~Samarati.
\newblock Access control of {SVG} documents.
\newblock In {\em Proceedings of DBSec 2002}, pages 219--230, 2002.

\bibitem{damiani00design}
E.~Damiani, S.~{De Capitani di Vimercati}, S.~Paraboschi, and P.~Samarati.
\newblock Design and implementation of an access control processor for {XML}
  documents.
\newblock {\em Computer Networks}, 33(1--6):59--75, 2000.

\bibitem{damiani02fine}
E.~Damiani, S.~{De Capitani di Vimercati}, S.~Paraboschi, and P.~Samarati.
\newblock A fine-grained access control system for {XML} documents.
\newblock {\em ACM Transactions on Information Systems Security},
  5(2):169--202, 2002.

\bibitem{damiani01controllingaccess}
E.~Damiani, P.~Samarati, S.~{De Capitani di Vimercati}, and S.~Paraboschi.
\newblock Controlling access to {XML} documents.
\newblock {\em IEEE Internet Computing}, 5(6):18--28, 2001.

\bibitem{divimercati96federated}
S.~{De Capitani di Vimercati} and P.~Samarati.
\newblock Access control in federated systems.
\newblock In {\em Proceedings of the 1996 workshop on new security paradigms},
  pages 87--99, Lake Arrowhead, California, United States, 1996. ACM Press.

\bibitem{heckman98applying}
M.~Heckman and K.~N. Levitt.
\newblock Applying the composition principle to verify a hierarchy of security
  servers.
\newblock In {\em {HICSS} (3)}, pages 338--347, 1998.

\bibitem{jackson:micromodels}
D.~Jackson.
\newblock Micromodels of software: Modeling and analysis with {A}lloy.
\newblock \texttt{http://sdg.lcs.mit.edu/alloy/reference-manual.pdf}.

\bibitem{jackson00automating}
D.~Jackson.
\newblock Automating first-order relational logic.
\newblock In {\em Proc. ACM SIGSOFT COnf. Foundations of Software Engineering},
  November 2000.

\bibitem{jackson:elements}
D.~Jackson and C.~A. Damon.
\newblock Elements of style: Analyzing a software design feature with a
  counterexample detector.
\newblock {\em IEEE Transactions on Software Engineering}, 22(7), July 1996.

\bibitem{jackson:alcoa}
D.~Jackson, I.~Schechter, and I.~Shlyakhter.
\newblock Alcoa: the {A}lloy constraint analyzer.
\newblock In {\em Proceedings of International Conference on Software
  Engineering}, Limerick, Ireland, June 2000. IEEE.

\bibitem{jackson01lightweight}
D.~Jackson and A.~Waingold.
\newblock Lightweight extraction of object models from bytecode.
\newblock In {\em IEEE Transactions on Software Engineering}, February 2001.

\bibitem{jajodia01multiplepolicies}
S.~Jajodia, P.~Samarati, M.~L. Sapino, and V.~S. Subrahmanian.
\newblock Flexible support for multiple access control policies.
\newblock {\em ACM Transactions on Database Systems}, 26(2):214--260, 2001.

\bibitem{jajodialogical}
S.~Jajodia, P.~Samarati, and V.~S. Subrahmanian.
\newblock A logical language for expressing authorizations.
\newblock In {\em Proceedings of the 1997 IEEE Symposium on Security and
  Privacy}, pages 31--42, Oakland, CA, USA, 1997. IEEE Press.

\bibitem{jajoida97unified}
S.~Jajodia, P.~Samarati, V.~S. Subrahmanian, and E.~Bertino.
\newblock A unified framework for enforcing multiple access control policies.
\newblock In {\em SIGMOD'97}, pages 474--485, Tucson, AZ, May 1997.

\bibitem{marinov:exhaustive}
D.~Marinov, A.~Andoni, D.~Danilinc, S.~Khurshid, and M.~Rinard.
\newblock An evaluation of exhaustive testing for data structures.
\newblock Technical Report MIT-LCS-TR-921, MIT CSAIL, 2003.

\bibitem{marinov02valloy}
D.~Marinov and S.~Khurshid.
\newblock {VA}lloy: Virtual functions meet a relational language.
\newblock In {\em 11th International Symposium of Formal Methods Europe (FME
  2002)}, Copenhagen, Denmark, July 2002.

\bibitem{naumovich:permission-analysis}
G.~Naumovich.
\newblock A conservative algorithm for computing the flow of permissions in
  {J}ava programs.
\newblock In {\em Proceedings of the International Symposium on Software
  Testing and Analysis ({ISSTA} '02)}, pages 33--43, July 2002.

\bibitem{nolte02filesystem}
T.~Nolte.
\newblock Exploring filesystem synchronization with lightweight modeling and
  analysis.
\newblock Master's thesis, MIT, August 2002.

\bibitem{samarati01policiesmodelsmechanisms}
P.~Samarati and S.~{De Capitani di Vimercati}.
\newblock {\em Foundations of Security Analysis and Design}, chapter~3, pages
  137--196.
\newblock Springer Verlag, 2001.

\bibitem{sandhu96audit}
R.~Sandhu and P.~Samarati.
\newblock Authentication, access control, and audit.
\newblock {\em ACM Computing Surveys}, 28(1):241--243, 1996.

\bibitem{sandhu93latticebased}
R.~S. Sandhu.
\newblock Lattice-based access control models.
\newblock {\em IEEE Computer}, 26(11):9--19, November 1993.

\balancecolumns
\bibitem{sandhu96rolebased}
R.~S. Sandhu, E.~J. Coyne, H.~L. Feinstein, and C.~E. Youman.
\newblock Role-based access control models.
\newblock {\em IEEE Computer}, 29(2):38--47, 1996.

\bibitem{sandhu94access}
R.~S. Sandhu and P.~Samarati.
\newblock Access control: Principles and practice.
\newblock {\em IEEE Communications Magazine}, 32(9):40--48, 1994 1994.

\bibitem{schaad:lightweight}
A.~Schaad and J.~Moffet.
\newblock A lightweight approach to specification and analysis of role-based access control extensions.
\newblock In {\em 7th ACM Symposium on Access Control Models and Technologies (SACMAT 2002)}, June 2002.

\bibitem{vaziri03heap}
M.~Vaziri and D.~Jackson.
\newblock Checking heap-manipulating procedures with a constraint solver.
\newblock In {\em TACAS'03}, Warsaw, Poland, 2003.

\bibitem{waingold01automated}
A.~Waingold.
\newblock Automated extraction of abstract object models.
\newblock Master's thesis, MIT, May 2001.

\bibitem{zao:rbac}
J.~Zao, H.~Wee, J.~Chu, and D.~Jackson.
\newblock {RBAC} schema verification using lightweight formal model and constraint analysis.
\newblock XXX

\bibitem{xacml}
{eXtensible Access Control Markup Language (XACML)} version 1.0.
\newblock OASIS Standard, February 2003.

\bibitem{XML}
Extensible markup language ({XML}) 1.0 (second edition), 2000.

\bibitem{xmlschema}
{XML Schema} part 2: Datatypes.
\newblock W3C Recommendation, May 2001.

\end{thebibliography}

\clearpage
\appendix
\section{XACML Representation of Voting Example and Alloy Translation}
\label{sec:xacml-repr-voting}

{ \small
\begin{verbatim}
<?xml version="1.0" encoding="UTF-8"?>
<Policy
  xmlns="urn:..."
  xmlns:xsi="...-instance"
  xmlns:md="http:.../record.xsd"
  PolicySetId="urn:example:policyid:1"
  RuleCombiningAlgId="urn:...:deny-overrides">
  <Target>
    <Subjects><AnySubject/></Subjects>
    <Resources><AnyResource/></Resources>
    <Actions>
      <Action>
        <ActionMatch MatchId="urn:...:string-equal">
          <AttributeValue DataType="...#string">
            vote
          </AttributeValue>
          <ActionAttributeDesignator
            AttributeId="urn:example:action"
            DataType="...#string"/>
        </ActionMatch>
      </Action>
    </Actions>
  </Target>
  <Rule RuleId="urn:example:ruleid:1" Effect="Deny">
    <Condition FunctionId="urn:...:integer-less-than">
      <Apply FunctionId="urn:...:integer-one-and-only">
        <SubjectAttributeDesignator
          AttributeId="urn:example:age"
          DataType="...#integer"/>
      </Apply>
      <AttributeValue DataType="...#integer">
        18
      </AttributeValue>
    </Condition>
  </Rule>
  <Rule RuleId="urn:example:ruleid:2" Effect="Deny">
    <Condition FunctionId="urn:...:boolean-equal">
      <Apply FunctionId="urn:...:boolean-one-and-only">
        <SubjectAttributeDesignator
          AttributeId="urn:example:voted-yet"
          DataType="...#boolean"/>
      </Apply>
      <AttributeValue DataType="...#boolean">
        True
      </AttributeValue>
    </Condition>
  </Rule>
  <Rule RuleId="urn:example:ruleid:3" Effect="Permit"/>
</Policy>
\end{verbatim}

}
\newpage

\section{Alloy Translation}
\label{sec:alloy-trans}

{ \small
\input{trimmedexample.als.tex}
}

\end{document}
